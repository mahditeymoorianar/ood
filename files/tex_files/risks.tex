ریسک‌ها را به این صورت دسته‌بندی  می‌کنیم:\\
\large \textbf{ریسک‌های امنیتی}
\begin{enumerate}[start = 1]
    \item \textbf{نقض حریم خصوصی کاربران}
       \\\textbf{توضیح: }
        ممکن است داده‌های شخصی کاربران نظیر اطلاعات کاربری، رایانامه یا شماره‌ی تلفن یا  … در دسترس افراد غیرمجاز (همچون هکرها) قرار بگیرد. همچنین ممکن است داده‌ها هنگام جابه‌جایی میان سرور و کلاینت، در صورت عدم استفاده از پروتکل‌های امنیتی مناسب، در اختیار هکرها قرار بگیرد.
       \\\textbf{راهکار: }
        با رمزنگاری داده‌های حساس و استفاده از پروتکل‌های  امنیتی مناسب می‌توان با این خطر مقابله کرد.
    \item \textbf{خطر از میان رفتن تمام یا بخشی از داده‌ها}
       \\\textbf{توضیح: }
        ممکن است بنابر هر دلیلی، به سرور آسیبی برسد و در نتیجه، تمام یا بخشی از داده‌های کاربران از دست برود. این داده‌ها ممکن است در شرایطی باعث ضرر مالی شود برای نمونه اگر داده‌های مربوط به ثبت اشتراک‌‌های  یک کاربر از میان برود، شخص علی‌رغم این که پیشتر پول پرداخته،‌ نمی‌تواند از اشتراک خود استفاده کند که این یک نمونه‌ای است که باعث ضرر مالی می‌شود. یا ممکن است اطلاعات یک کانال که تعداد زیادی عضو دارد بپرد و روشن است که این باعث ضرر مالی بزرگی می‌تواند بشود.
       \\\textbf{راهکار: }
        بهتر است در بازه‌های زمانی مناسبی، از همه‌ی داده‌های روی سرور، پشتیبان بگیریم.
    \item \textbf{دسترسی غیر مجاز به محتوای پولی}
       \\\textbf{توضیح: }
        ممکن است با روش‌هایی ناشی از نقص‌های امنیتی، برخی افراد غیرمجاز بتوانند بی آنکه اشتراک بخرند، به محتواهای پولی کانال‌ها دسترسی بیابند. این اتّفاق می‌تواند باعث نارضایتی تولیدکنندگان محتوا و در نتیجه از میان رفتن شبکه‌ی اجتماعی قاصدک شود.
       \\\textbf{راهکار: }
        باید همه‌ی حفره‌های  امنیتی سامانه را به دقّت شناخت و با روش‌های مناسب مانند به‌کارگیری پروتکل‌های امنیتی مناسب، بررسی ایمنی سرورها در برابر نرم‌افزارهای مخرّب و … و انجام آزمون‌های نرم‌افزاری لازم، جلوی این موضوع را گرفت.
    \item \textbf{دسترسی  افراد غیرمجاز به اطلاعات کانال‌ها}
       \\\textbf{توضیح: }
        ممکن است افرادی که مدیر کانال نیستند بتوانند با روش‌هایی ناشی از نقص امنیتی، به اطلاعات کانال‌ها دسترسی بیابند و در کانال‌ها پست بگذارند یا تغییراتی در کانال ایجاد کنند که به روشنی نامطلوب است. همچنین ممکن است برخی مدیران که مالک نیستند بتوانند با دسترسی به امکانات مالک، فعالیت‌های غیر مجازی مانند افزایش سهم خود انجام دهند.
       \\\textbf{راهکار: }
        باید همه‌ی حفره‌های  امنیتی سامانه را به دقّت شناخت و با روش‌های مناسب مانند به‌کارگیری پروتکل‌های امنیتی مناسب، بررسی ایمنی سرورها در برابر نرم‌افزارهای مخرّب و … و انجام آزمون‌های نرم‌افزاری لازم، جلوی این موضوع را گرفت.
        \item \textbf{امکان بارگذاری محتوای مخاطره‌آمیز مانند فایل‌های ویروسی و مخرّب}
       \\\textbf{توضیح: }
        ممکن است هنگام بارگذاری فایل‌های چندرسانه‌ای مانند ویدیو، فایل مورد نظر مخرّب یا ویروسی باشد و کاربرانی که آن را دریافت می‌کنند یا حتّی سرور که فایل روی آن قرار می‌گیرد، به خطر بیفتد.
       \\\textbf{راهکار: }
        برای جلوگیری از این موضوع می‌توان آنتی‌ویروس خوبی روی سرور نصب کرد تا جلوی فایل‌های مخرّب را بگیرد و با تشخیص آنها، آنها را حذف کند. همچنین در راستای ایمنی بیشتر، می‌توان در کلاینت نیز این موضوع را بررسی اوّلیّه‌ای کرد تا اگر فایل مخرّب بود، از فرستادن آن به سرور جلوگیریم.
    \item \textbf{خطا در انتساب نقش‌های چهارگانه}
       \\\textbf{توضیح: }
        ممکن است سامانه در انتساب نقش‌های چهارگانه (مالک، مدیر، کاربر ویژه، کاربر عادّی) دچار اشتباه شود و در نتیجه دسترسی‌های نادرستی به افراد داده شود. برای مثال به کاربری که اشتراک دارد به جای «کاربر ویژه»، «کاربر عادّی» نسبت داده شود و در نتیجه دسترسی به محتوای پولی نداشته باشد که این به روشنی نامطلوب است.
       \\\textbf{راهکار: }
        با انجام آزمون‌های کیفی مورد نیاز، باید از صحّت عملکرد سامانه در سناریوهای گوناگون مطمئن شد. همچنین باید امنیت سیستم تأمین شود.
    \item \textbf{خطا در محاسبه‌ی تعرفه‌ها}
       \\\textbf{توضیح: }
        ممکن است هنگام نمایش تعرفه‌ها و پرداخت آنها، مقدار اشتباهی پول از کاربر گرفته شود. این مشکل ممکن است به علّت ایراد در سامانه باشد یا به خاطر خرابکاری عامدانه برای کاستن مقدار پول پرداخت شده باشد.
       \\\textbf{راهکار: }
        همچنان راهکارمان برای مقابله با این خطر، انجام آزمون‌های کیفی مناسب و مقابله با نواقص امنیتی و نفوذپذیری سامانه است.
    \item \textbf{خطا در محاسبه‌ی مدت اشتراک}
       \\\textbf{توضیح: }
        ممکن است مدّت اشتراک کاربران ویژه اشتباه محاسبه شود برای نمونه اگر اشتراک کاربری کمتر از مقدار واقعی محاسبه شود و در نتیجه زودتر از به پایان رسیدن آن، جلوی کاربر از دیدن محتویات پولی، گرفته شود یا اشتراک بیشتر از مدّت واقعی محاسبه شود و پس از پایان آن همچنان اجازه‌ی مشاهده‌ی محتویات پولی داده شود. این موضوع ممکن است به دلیل باگ در سامانه یا نواقص امنیتی و سوء استفاده‌های عامدانه انجام شود.
       \\\textbf{راهکار: }
        همچنان راهکارمان برای مقابله با این خطر، انجام آزمون‌های کیفی مناسب و مقابله با نواقص امنیتی و نفوذپذیری سامانه است.
\end{enumerate}

\large \textbf{ریسک‌های مربوط به محدوده و نیازمندی‌ها}
\begin{enumerate}[start = 9]       
    \item \textbf{روشن نبودن خواسته‌های مشتری یا درک نادرست نیازمندی‌ها توسط تیم ایجاد}
       \\\textbf{توضیح: }
        ممکن است برخی خواسته‌های مشتری به طور کامل روشن نباشد یا توسط تیم ایجاد به درستی فهمیده نشود. این ممکن است به این خاطر باشد که انتقال خواسته‌های مشتری به تیم ایجاد به درستی صورت نگرفته یا آن که برخی از خواسته‌های مشتری، برای خودش هم به طور کامل مشخّص نبوده و با جلوتر رفتن برایش روشن می‌شود.
       \\\textbf{راهکار: }
        پیش از آغاز پروژه، نیازمندی‌ها تحلیل شوند و به تأیید مشتری برسند. همچنین باید مدام با مشتری ارتباط داشت تا جلوی بروز مشکلات را گرفت یا مشکلات از این دست را هرچه زودتر تشخیص داده و حل کرد. نیز باید قابلیت اصلاح‌پذیری و maintenance پروژه بالا باشد تا بتوان در صورت بروز نیازمندی‌های جدید یا تغییر خواسته‌های مشتری، اصلاحات لازمه را به سادگی پیاده ساخت.
    \item \textbf{بزرگ‌شدن بیش از حد پروژه}
       \\\textbf{توضیح: }
        ا جلو رفتن پروژه ممکن است نیازمندی‌های جدید بروز یابند و خواسته‌های مشتری بیشتر شود و این ممکن است به تدریج باعث شود پروژه بیش از اندازه بزرگ شود و پیاده‌سازی آن برای تیم ایجاد، با شرایط و امکانات موجود مشکل یا ناممکن شود.
       \\\textbf{راهکار: }
        برای مقابله با این ریسک، راهکارهایی در نظر گرفته‌ایم از جمله تعریف دقیق نیازمندی‌ها، توافق با مشتری بر سقف قیمت و زمان و اندازه.
    \end{enumerate}

\large \textbf{ریسک‌های مربوط به زمان‌بندی}
\begin{enumerate}[start = 11] 
    \item \textbf{تخمین‌های زمانی نادرست برای انجام تسک‌ها}
       \\\textbf{توضیح: }
        ممکن است برخی  از کارها در زمان پیش‌بینی شده تمام نشوند. در اینجا اتفاقات پیش‌بینی نشده مد نظر نیستند و مسئله تخمین نادرست از زمان مورد نیاز برای انجام کارها است.
       \\\textbf{راهکار: }
        باید هنگام برنامه‌ریزی، کوشید که تا جای ممکن با دقّت بالایی زمان‌بندی پروژه را پیش‌بینی کرد و برنامه‌ریزی دقیقی داشت و در این کار باید از اشخاص با تجربه کمک گرفت. با این وجود همچنان این مشکل ممکن است که باید در آن صورت بتوان برنامه‌ریزی را طوری تغییر داد که کمترین آسیب به روند پروژه وارد آید و پروژه شکست نخورد. همچنین می‌توان هنگام اعلام ددلاین به مشتری، مقداری ددلاین را بیشتر از زمان پیش‌بینی‌شده اعلام کرد تا در صورت پیش‌آمدن این مشکل، به مشتری نگرانی‌ای تحمیل نشود.
    \item \textbf{اعلام دیرهنگام بسترهای مجاز برای انجام پروژه}
       \\\textbf{توضیح: }
        این مورد ممکن است که به تاخیر در یادگیری یک بستر جدید منجر شود که  
        در نهایت به تاخیر در تحویل پروژه بیانجامد.    
       \\\textbf{راهکار: }
        پیدا کردن روشی برای انتقال حداکثری موارد تولید شده به بستر جدید،استفاده از بستری مجازی  نزدیک به \lr{contex} پروژه و   بستر هایی که‌قبلا برای پروژه های مشابه استفاده شده
    \item \textbf{ناتوانی در تحویل فازهای پروژه در ددلاین مشخص شده}
       \\\textbf{توضیح: }
        به دلایل مختلف مانند کوتاهی اعضای گروه در انجام وظایف تقسیم بندی شده فاز های پروژه در ددلاین مشخص شده تحویل داده نمی شود
       \\\textbf{راهکار: }
        شناسایی  مهارت های های اعضا و محول کردن تسک هایی در جهت مهارت ها 
\end{enumerate}

\large \textbf{ریسک‌های مربوط به مدیریت تغییرات}
\begin{enumerate}[start = 14] 

    \item \textbf{نبود قابلیت Maintainability }
       \\\textbf{توضیح: }
        در صورت نبود Maintability شاهد افت کیفیت خواهیم بود
        \\\textbf{راهکار: }
        اصول ظراحی شی گرا را رعایت کنیم و پیاده سازی خوبی داشته باشیم
    \end{enumerate}

\large \textbf{ریسک‌های مربوط به ارتباطات}
\begin{enumerate}[start = 15] 
    \item \textbf{کمرنگ شدن ارتباط دستیاران با تیم}
       \\\textbf{توضیح: }
        فعالیت دستیاران درس در طول ترم کمرنگ شود و بازخوردها به موقع ارائه نشوند.
       \\\textbf{راهکار: }
        پیگیری از دستیاران با ایمیل و راه های ارتباطی دیگر
\end{enumerate}

\large \textbf{ریسک‌های مربوط به نیروی انسانی}
\begin{enumerate}[start = 16] 
        \item \textbf{خروج عضوی از تیم پیش از پایان یافتن پروژه            }
       \\\textbf{توضیح: }
        یکی از اعضای تیم، پیش از پایان یافتن پروژه ممکن است جهت تحصیل در خارج از کشور مهاجرت کند و از تیم خارج شود.
       \\\textbf{راهکار: }
        سه راه حل برای این ریسک قابل تصور است که به اقتضای موقعیت، یک باید انتخاب شود: 
        ١. انجام تسک ها به صورت فشرده ای باشد تا پیش از مهاجرت عضو، پروژه پایان یافته باشد. 
        ٢. در حین انجام پروژه، بقیه اعضای تیم در جریان جزییات تسک هایی که آن فرد انجام مͬ دهد قرار گیرند تا بتوانند در صورت لزوم ادامه ی تسک های وی را انجام دهند.
         ٣. امکان برقراری ارتباط از راه دور با عضو خارج شده میسر باشد تا همکاری بدین صورت انجام شود.
        
    \item \textbf{پیش‌آمدن مشکل‌های غیر مترقبه}
       \\\textbf{توضیح: }
        برای نمونه، امتحان های درسی ممکن است منجر به در دسترس نبودن یک یا چند عضو در یک بازه‌ی زمانی بحرانی شود.
       \\\textbf{راهکار: }
        در هنگام تخمین های زمانی، باید این موارد تا حد امکان در نظر گرفته شوند.
    \item \textbf{پیش‌آمدن مشکل برای اعضای تیم در زمان‌های خاص}
       \\\textbf{توضیح: }
        پروژه های درس های دیگر ممکن است باعث شود که برنامه‌ریزی‌ها با مشکل مواجه شوند که خود می تواند منجر به کاهش کیفیت محصولات ترخیص شود.
       \\\textbf{راهکار: }
        برنامه ریزی مناسب و مدیریت اعضای تیم
\end{enumerate}

\large \textbf{ریسک‌های فنّی}
\begin{enumerate}[start = 19] 
        \item \textbf{بروز مشکلات سخت‌افزاری و نرم‌افزاری}
       \\\textbf{توضیح: }
        برای نمونه خراب شدن کامپیوتر/لپ‌تاپ یکی از اعضای تیم، بروز ایراد طولانی مدت در اتصال اینترنت یا بروز اشکال در سرور سایت Trello یا Github 
       \\\textbf{راهکار: }
        برای مقابله با مشکلات نرم افزاری مربوط به شبکه، لازم است سعی شود همواره یک رونوشت از آخرین محصولات به صورت offline موجود باشد.
    \item \textbf{نیاز به یادگیری مهارت‌های جدید}
       \\\textbf{توضیح: }
        ممکن است لازم شود اعضای تیم موارد جدیدی را یاد بگیرند که باعث نادقیق بودن تخمین های زمانی شود
       \\\textbf{راهکار: }
        می‌توان کوشید تا جای ممکن از روش‌هایی استفاده‌کرد که اعضای تیم مهارتش را دارند و در صورت نیاز از آموزش‌های مناسب و کمک‌گرفتن از افرادی که آن مهارت‌ها را بلد هستند بهره جست.
    \item \textbf{نامناسب‌بودن ابزارهای CASE مورد استفاده}
       \\\textbf{توضیح: }
        ممکن است ابزارهای CASE مورد استفاده، بعضی از نمودارهایی را که لازم است تولید شوند پشتیبانی نکند. در نتیجه هماهنگ کردن نمودارهای مختلف با یکدیگر ممکن است مشکل‌ساز شود. همچنین، ممکن است اعضای تیم ایجاد شناخت دقیقی از ابزار CASE مورد استفاده نداشته باشند و انتظاری داشته باشند که با این ابزار قابل رفع نباشد.
       \\\textbf{راهکار: }
        تشویق اعضای تیم به مطالعه‌ی دقیق مستندات و راهنماهای ابزارهای مختلف CASE پیش از انتخاب و استفاده تا حد خوبی این ریسک را کاهش می‌دهد.
    \item \textbf{عدم رضایت مشتری از محصول نهایی}
       \\\textbf{توضیح: }
        در صورتی که محصول نهایی در آزمون پذیرش رد شود، هر چقدر هم که از استانداردهای کیفی تبعیت کرده باشد، پروژه شکست خورده اعلام خواهد
       \\\textbf{راهکار: }
        با تحلیل دقیق، تعامل بیشتر با مشتری و اعمال بازخوردهای هر فاز، می توان ریسک مورد قبول واقع نشدن محصول نهایی را کاهش داد.
    \item \textbf{مشکل در سامانه‌ی پرداختن پول}
       \\\textbf{توضیح: }
        مشکلات در زمینه پرداخت مثل از کار افتادن درگاه‌های بانکی
       \\\textbf{راهکار: }
        راه جایگزین به جای پرداخت پول آنلاین مثل پرداخت حضوری در عابر بانک‌ها و هماهنگی با پشتیبانی، یا صبر کردن تا درست‌شدن یا استفاده از رمزارزها
    \item \textbf{ناکافی بودن فضای ذخیره‌سازی سرور برای نگه‌داری داده‌های سیستم}
       \\\textbf{توضیح: }
        فضای ذخیره‌سازی درنظرگرفته‌شده برای نگه‌داری داده‌های مورد نیاز کافی نیست.
       \\\textbf{راهکار: }
        افزودن  سرورهای ذخیره‌سازی با در نظر گیری بودجه
    \item \textbf{مقیاس‌پذیری و توانایی پاسخ‌گویی به شمار زیادی کاربر به طور همزمان به‌شیوه‌ای کارا        }
       \\\textbf{توضیح: }
        در بعضی موارد لازم  تعداد زیادی از کاربران به صورت همزمان خدمت داده شوند و ترافیک و بار سیستم به شدت بالا می‌رود و لازم است تدابیری برای آن اندیشیده شود.
       \\\textbf{راهکار: }
        داشتن سرور پشتیبان
    \item \textbf{تأمین‌ هزینه‌ی سرورها و \dots}
       \\\textbf{توضیح: }
        برای پیاده سازی سامانه احتیاج به سرور قدرتمند داریم
       \\\textbf{راهکار: }
        خرید سرورها میتواند از سمت مشتری به‌جای تیم ایجاد انجام شود. همچنین می‌توان برای پروژه اسپانسر یافت.
    \item \textbf{پرداختن حقوق کارکنان}
       \\\textbf{توضیح: }
        در صورت دیرکرد در پرداخت حقوق کارکنان ممکن است پروژه پیش نرود و به تاخیر بیافتد 
       \\\textbf{راهکار: }
        گرفتن اسپانسر یا دادن فیش حقوقی به کارمندان
\end{enumerate}

\zirtitr{اولویت‌بندی ریسک‌ها}
        در این جا ریسک‌ها را بر اساس اولویت در ۷ دسته قرار داده‌ایم
        امّا برای ریسک‌های در یک دسته‌ی اولویتی، از نظر اولویت
        تفاوتی قائل نشده‌ایم. در این اولویت‌بندی، کوشیده‌ایم
        به معیارهای نقض حریم خصوصی و مشکلات امنیتی و مسائل مالی
        اهمّیّت بالاتری بدهیم و نیز با توجّه به شدّت و احتمال موارد
        و سختی حل کردن آنها، مرتّب کنیم.

        \zirzirtitr{اولویت بسیار بالا}
        \begin{itemize}
            \item نقض حریم خصوصی کاربران
            \item دسترسی غیرمجاز به محتوای پولی
            \item خطاهای پرداخت
            \item خطر از میان رفتن تمام یا بخشی از داده‌ها
        \end{itemize}

        \zirzirtitr{اولویت بالا}
        \begin{itemize}
            \item دسترسی افراد غیرمجاز به اطلاعات کانال‌ها
            \item امکان بارگذاری محتوای مخاطره‌آمیز مانند فایل‌های ویروسی و مخرّب
            \item خطا در انتساب نقش‌های چهارگانه
            \item خطا در محاسبه‌ی تعرفه‌ها
            \item خطا در محاسبه‌ی مدت اشتراک
        \end{itemize}

        \zirzirtitr{اولویت نسبتاً بالا}
        \begin{itemize}
            \item روشن‌نبودن محدوده‌ی پروژه و درک نادرست نیازمندی‌ها توسط تیم ایجاد
            \item عدم رضایت مشتری از محصول نهایی
        \end{itemize}

        \zirzirtitr{اولویت متوسط}
        \begin{itemize}
            \item نبود قابلیت Maintability ودر نتیجه افت کیفیت
            \item ناکافی بودن فضای ذخیره‌سازی سرور برای نگه‌داری داده‌های سیستم
            \item نیاز به یادگیری مهارت‌های جدید
            \item مقیاس‌پذیری و توانایی پاسخ‌گویی به شمار زیادی کاربر به طور همزمان به‌شیوه‌ای کارا
        \end{itemize}

        \zirzirtitr{اولویت نسبتاً پایین}
        \begin{itemize}
            \item خروج عضوی از تیم پیش از پایان یافتن پروژه
            \item پیش‌آمدن مشکل‌های غیر مترقبه
            \item پیش‌آمدن مشکل برای اعضای تیم در زمان‌های خاص
        \end{itemize}

        \zirzirtitr{اولویت پایین}
        \begin{itemize}
            \item بروز مشکلات سخت‌افزاری و نرم‌افزاری
            \item نامناسب بودن ابزارهای CASE مورد استفاده
            \item مشکل در سامانه‌ی پرداختن پول (مثلاً از کار افتادن درگاه بانکی)
        \end{itemize}

        \zirzirtitr{اولویت پایین‌تر}
        \begin{itemize}
            \item تأمین‌ هزینه‌ی سرورها و \dots
            \item پرداختن حقوق کارکنان
            \item بزرگ‌شدن بیش از حد پروژه
            \item تخمین‌های زمانی نادرست برای انجام تسک‌ها
            \item اعلام دیرهنگام بسترهای مجاز برای انجام پروژه
            \item ناتوانی در تحویل فازهای پروژه در ددلاین مشخص شده
            \item ریسک‌های مربوط به ارتباطات
            \item کمرنگ شدن ارتباط دستیاران با تیم
        \end{itemize}
