\zirtitr{معماری کلی}
از معماری Service-Oriented Architecture استفاده می‌کنیم این معماری یک رویکرد برای ساخت سیستم‌های نرم‌افزاری است که بر اساس سرویس سازماندهی شده‌اند. در این معماری، هر سرویس به عنوان یک واحد مستقل و قابل استفاده در سیستم تعریف می‌شود که قابل استفاده بودن آن با سرویس‌های دیگر را فراهم می‌کند.

\zirtitr{معماری سرور }
معماری Model-View-Template (MVT) یک الگوی طراحی برای توسعه وب است که در فریم‌ورک Django استفاده می‌شود. این معماری شامل سه بخش اصلی است:

\begin{enumerate}
\item مدل (Model): این بخش شامل داده‌های برنامه و روابط آن‌ها با یکدیگر است. مدل ها به عنوان نقطه شروع برای تعریف داده ها و روابط آن ها در پایگاه داده استفاده می شود.

\item نمایش (View): این بخش شامل کدهای لازم برای پردازش درخواست کاربر و نمایش صفحات وب است. در این قسمت، کدهای لازم برای پاسخ به درخواست های HTTP، پردازش داده های فرستاده شده توسط کاربر، و نمایش صفحات HTML به کاربران تولید می شود.

\item قالب (Template): این بخش شامل قالب های HTML است که به عنوان نقطه خروجی برنامه استفاده می شود. قالب ها حاوی کدهای HTML، CSS و JavaScript هستند که برای نمایش داده ها به کاربران استفاده می شوند.
\end{enumerate}
