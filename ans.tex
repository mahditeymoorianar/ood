\documentclass{report}
\usepackage{graphicx}
\usepackage[left = 20mm, right = 20mm, top = 20mm, head = 10mm]{geometry}
\usepackage{subcaption}
\usepackage{fancyhdr}
\usepackage{float}
\usepackage{listings}
\usepackage{listings}
\usepackage{color}
\usepackage{xcolor}
\usepackage{amsmath}
\usepackage{mathtools}
\usepackage{amssymb}
\usepackage{hyperref}
\usepackage{etoolbox}
\usepackage{multirow}
\usepackage{colortbl}
\usepackage{enumitem}
\usepackage{longtable}

%%
%% HERE IS THE MOST IMPORTANT INFORMATION ABOUT
%%      THIS DOCUMENT
%%
\newcommand{\subjectname}{طرّاحی سیستم‌های شئ‌گرا}
\newcommand{\profname}{دکتر رامسین}
% \newcommand{\assignmentNo}{7}
\newcommand{\dateof}{تـــابســــتان ۱۴۰۲}
\hypersetup{
    colorlinks=true,
    linkcolor=blue,
    citecolor=darkblue,
    filecolor=magenta,      
    urlcolor=purple,
    pdftitle={مستند فاز ۱ پروژه‌ی درس طرّاحی سیستم‌های شئ‌گرا},
    % pdftitle={تمرین \assignmentNo \subjectname},
    pdfauthor={مهدی تیموری انار},
    % pdfkeywords={, شئ‌گرا, شریف, دانشگاه, تکلیف},
    pdfkeywords={تکلیف, دانشگاه, شریف, شئ‌گرا, , assignment, university, ood, object oriented, computer, Sharif}
    pdfpagemode=FullScreen,
}
% \usepackage{algorithm}
% \usepackage{algorithmic}
\usepackage[linesnumbered, ruled]{algorithm2e}
% \include{pythonlisting}
% \usepackage{mismath}
\usepackage[utf8]{inputenc}
\usepackage{xepersian}
\pagestyle{fancy}
% \symbol{"FDFD}\\\vspace{2mm}
\fancyhead[C]{
    \includegraphics*[width = 0.08\textwidth]{logo.png}
}
\fancyhead[R]{مهدی تیموری انار، آیناز رفیعی، حامد عبدی، پرهام عسکرزاده\\\dateof}
\fancyhead[L]{فاز ۳ پروژه‌ی درس طرّاحی شئ‌گرا\\شبکه‌ی اجتماعی قاصدک}
\fancyheadoffset{10mm}
\renewcommand{\headrulewidth}{0.4pt}
\renewcommand{\footrulewidth}{0.4pt}
% \headrulewidth{0.4pt}
% \footrulewidth{0 pt}
\fancyfoot[C]{رویه‌ی \thepage}
\settextfont{Amiri}
\setmathdigitfont[]{Amiri}
\setlatinsansfont[]{Amiri}
\setlatintextfont[]{Amiri}
% \setlatintextfont{Consolas}
% \setlatinsansfont{Consolas}
\author{\large مهدی تیموری انار - ۹۹۱۰۱۳۵۴\\
\large آیناز رفیعی - ۹۸۱۷۰۸۱۶\\
\large حامد عبدی - ۹۶۱۰۹۷۸۲\\
\large پرهام عسکرزاده - ۹۸۱۷۰۹۳۵
}
\date{\dateof}

% Label format
\DeclareCaptionLabelFormat{custom}
{%
      نگاره‌ی #2
}
% Separator style
% \DeclareCaptionLabelSeparator{custom}{--}
% Caption format    
% \DeclareCaptionFormat{custom}
% {%
%     #1#2\small #3
% }
\renewcommand{\figurename}{نگاره‌ی}
% \renewcommand{\algorithmcfname}[][]{الگوریتم}
% \renewcommand{\lstlistingname}{\begin{persian}پاره‌کد\end{persian}}
\renewcommand{\lstlistingname}{code}
% \captionsetup
% {
%     % format=custom,%
%     labelformat=custom,%
%     % labelsep=custom
% }
%
\definecolor{customgreen}{rgb}{0,0.6,0}
\definecolor{customgray}{rgb}{0.5,0.5,0.5}
\definecolor{purple}{rgb}{0.31,0,0.31}
\definecolor{pink}{rgb}{0.85,0.21,0.41}
\definecolor{custommauve}{rgb}{0.6,0,0.8}
\definecolor{darkblue}{rgb}{0, 0, 0.8}
\definecolor{codeblockbg}{rgb}{0.85, 0.95, 1}

\newtoggle{InString}{}% Keep track of if we are within a string
\togglefalse{InString}% Assume not initally in string

\newcommand*{\ColorIfNotInString}[1]{\iftoggle{InString}{#1}{\color{blue}#1}}%
\newcommand*{\ProcessQuote}[1]{#1\iftoggle{InString}{\global\togglefalse{InString}}{\global\toggletrue{InString}}}%

\lstset{ 
    basicstyle=\small,        % the size of the fonts that are used for the code
    breaklines=true,                 % sets automatic line breaking
    commentstyle=\color{customgreen},    % comment style
    firstnumber=1,                % start line enumeration with line 1000
    frame=single,	                   % adds a frame around the code
    keepspaces=true,                 % keeps spaces in text, useful for keeping indentation of code (possibly needs columns=flexible)
    keywordstyle=\color{blue},       % keyword style
    numbers=left,                    % where to put the line-numbers; possible values are (none, left, right)
    numbersep=10pt,                   % how far the line-numbers are from the code
    numberstyle=\tiny\color{customgray}, % the style that is used for the line-numbers
    rulecolor=\color{black},         % if not set, the frame-color may be changed on line-breaks within not-black text (e.g. comments (green here))
    showspaces=false,                % show spaces everywhere adding particular underscores; it overrides 'showstringspaces'
    showstringspaces=false,          % underline spaces within strings only
    showtabs=false,                  % show tabs within strings adding particular underscores
    stepnumber=1,                    % the step between two line-numbers. If it's 1, each line will be numbered
    stringstyle=\color{custommauve},     % string literal style
    tabsize=2,	                   % sets default tabsize to 2 spaces
    title=\lstname,                   % show the filename of files included with \lstinputlisting; also try caption instead of title
    % literate=
    % {0}{{{\ProcessDigit{0}}}}1
    % {1}{{{\ProcessDigit{1}}}}1
    % {2}{{{\ProcessDigit{2}}}}1
    % {3}{{{\ProcessDigit{3}}}}1
    % {4}{{{\ProcessDigit{4}}}}1
    % {5}{{{\ProcessDigit{5}}}}1
    % {6}{{{\ProcessDigit{6}}}}1
    % {7}{{{\ProcessDigit{7}}}}1
    % {8}{{{\ProcessDigit{8}}}}1
    % {9}{{{\ProcessDigit{9}}}}1
    % {<=}{{\(\leq\)}}1
    literate=%
    {"}{{{\ProcessQuote{"}}}}1% Disable coloring within double quotes
    {'}{{{\ProcessQuote{'}}}}1% Disable coloring within single quote
    {0}{{{\ColorIfNotInString{0}}}}1
    {1}{{{\ColorIfNotInString{1}}}}1
    {2}{{{\ColorIfNotInString{2}}}}1
    {3}{{{\ColorIfNotInString{3}}}}1
    {4}{{{\ColorIfNotInString{4}}}}1
    {5}{{{\ColorIfNotInString{5}}}}1
    {6}{{{\ColorIfNotInString{6}}}}1
    {7}{{{\ColorIfNotInString{7}}}}1
    {8}{{{\ColorIfNotInString{8}}}}1
    {9}{{{\ColorIfNotInString{9}}}}1
    }
%
\title{
    \vspace{-40mm}
    \symbol{"FDFD}\\
    \vspace{20mm}
    {\includegraphics[width = 300pt]{logo.png}}\\
    \vspace{10mm}
    % {\huge  \assignmentNتمرین سریo \subjectname}\\
    {\huge فاز ۳ پروژه‌ی طرّاحی سیستم‌های شئ‌گرا}\\
    \vspace{5mm}
    {\large \subjectname - \profname}\\
    %TODO:
    {\normalsize پیوست: ندارد.}
    \vspace{-5mm}
}
% \defpersianfont\titr[Scale=1.5]{Amiri}

% \defpersianfont\fehrest[Scale=1.2]

% \defpersianfont\fehrestTitle[Scale=3.0]

% \defpersianfont\fehrestContent[Scale=1.2]

% \newcounter{question-count}
% % \setcounter{question-count}{1}
% \newcommand{\question}[1]
% {	
% 	\stepcounter{question-count}
% 	\section*{{\titr سؤال #1. }}
	
% 	\addcontentsline{toc}{section}{{\fehrestContent سؤال \arabic{1}. }}
	
%     % #1
% }

\lstdefinelanguage{JavaScript}{
    backgroundcolor=\color{codeblockbg},
    basicstyle=\ttfamily\normalsize,
  keywords={break, toString, bind, case, catch, continue, debugger, default, delete, do, else, false, finally, for, function, if, in, instanceof, new, null, return, switch, this, throw, true, try, typeof, var, void, while, with},
  morecomment=[l]{//},
  morecomment=[s]{/*}{*/},
  morestring=[b]',
  morestring=[b]",
  ndkeywords={assert before, after, beforeEach, afterEach, skip, 
    retries, only, it, equal, describe, require, class, export, boolean, throw, implements, import, this},
  keywordstyle=\color{blue}\bfseries,
  ndkeywordstyle=\color{pink}\bfseries,
  identifierstyle=\color{black},
  commentstyle=\color{purple}\ttfamily,
  stringstyle=\color{customgreen}\ttfamily,
  sensitive=true
}

\newcommand{\digitstyle}{\color{red}}
\newcommand{\ProcessDigit}[1]
{%
  \ifnum\lst@mode=\lst@Pmode\relax%
   {\digitstyle #1}%
  \else
    #1%
  \fi
}
\lstdefinelanguage{mips}{
    backgroundcolor=\color{codeblockbg},
    basicstyle=\ttfamily\normalsize,
  keywords={lw, sw, add, sub, bne, nop},
  morecomment=[l]{\#},
  morecomment=[s]{/*}{*/},
  morestring=[b]',
  morestring=[b]",
  ndkeywords={\$,\#,(, ), 0, 1, 2, 3, 4, 5, 6, 7, 8, 9, Loop,
  \$1, \$2, \$3, \$4, \$5, \$6, \$7, \$8, \$9, \$0},
  extendedchars=true,
  numbersep=5pt,
  numbers=left,
  keywordstyle=\color{pink}\bfseries,
  ndkeywordstyle=\color{blue}\bfseries,
  identifierstyle=\color{black},
  commentstyle=\color{customgreen}\ttfamily,
  stringstyle=\color{customgreen}\ttfamily,
  numberstyle=\color[rgb]{0.805, 0.142, 0.73},
%   digitstyle=\color[rgb]{0.805, 0.142, 0.73},
  sensitive=true,
}


\newcommand{\question}[1]{
    % \pagebreak
    \section*{{سؤال #1}}
    \hspace*{3mm}
    \addcontentsline{toc}{section}{{سؤال #1}}
}

\newcommand{\questionpart}[1]{
    \section*{{\large #1)}}
    % \vspace*{-9mm}
    \hspace*{3mm}
    \addcontentsline{toc}{subsection}{{#1}}
}

\newcounter{titr-count}
\newcounter{zirtitr-count}
\newcounter{zirzirtitr-count}
\setcounter{titr-count}{-1}
\setcounter{zirtitr-count}{0}
\setcounter{zirzirtitr-count}{0}

\newcommand{\titr}[1]{
    \setcounter{zirtitr-count}{0}
    \stepcounter{titr-count}
    \section*{\arabic{titr-count}- #1}
    \addcontentsline{toc}{section}{\arabic{titr-count}- #1}
    \hspace*{3mm}
}
\newcommand{\zirtitr}[1]{
    \stepcounter{zirtitr-count}
    \subsection*{\arabic{titr-count}-\arabic{zirtitr-count}- #1}
    \addcontentsline{toc}{subsection}{\arabic{titr-count}-\arabic{zirtitr-count}- #1}
    \hspace*{3mm}
}
\newcommand{\zirzirtitr}[1]{
    \stepcounter{zirzirtitr-count}
    \subsubsection*{\arabic{titr-count}-\arabic{zirzirtitr-count}-\arabic{zirzirtitr-count}- #1}
    \addcontentsline{toc}{subsubsection}{\arabic{titr-count}-\arabic{zirzirtitr-count}-\arabic{zirzirtitr-count}- #1}
    \hspace*{3mm}
    \vspace*{-8mm}
}
\newcommand{\expected}[1]{
    \mathbb{E}[#1]
}
\newcommand{\var}[1]{
    \textrm{Var}(#1)
}
\newcommand{\negare}[1]{
    نگاره‌ی \ref{#1}
}
\renewcommand{\listfigurename}{فهرست نگاره‌ها}
\renewcommand{\lstlistingname}{\rl{پاره‌کد}}
\renewcommand{\lstlistlistingname}{فهرست کدها}
\begin{document}
    \maketitle
    \pagebreak
    \tableofcontents
    \pagebreak
    \listoffigures
    % \pagebreak
    % \lstlistoflistings
    % \pagebreak
    % \listoftables
    \pagebreak
    \titr{تعریف زیرسیستم‌ها}
        \begin{itemize}
            \item \textbf{ناحیه‌ی کاربری: } 
                وظیفه‌ی این زیرسیستم، مدیریت اطلاعات کاربران، ثبت نام  و ورود، لیست کانال‌های عضو شده و … است.
            \item \textbf{مدیریت کانال: }
                این زیرسیستم وظایفی همچون ایجاد کانال و نهادن محتوا در کانال، انتساب نقش‌های چهارگانه و امکان تعیین مدیر توسط مالک کانال و … را بر عهده دارد.
            \item \textbf{مدیریت اشتراک و پرداخت ها: } 
                امکان تعریف تعرفه‌های اشتراک و انواع تعرفه‌ها، امکان خریدن اشتراک‌ها، امکان تعیین سهم مدیر توسط مالک و در کل مسائل مربوط به امور مالی و اشتراک‌ها را بر عهده دارد.
            \item \textbf{مدیریت دسترسی  و نمایش محتوا: }
                این زیرسیستم وظایف مربوط به بررسی دسترسی افراد به محتوا را بر عهده دارد از جمله امکان مشاهده‌ی پست‌های رایگان توسط همه‌ی کاربران کانال، امکان مشاهده‌ی عنوان و خلاصه‌ی محتواهای پولی برای اعضای عادی و امکان مشاهده‌ی کل پست‌ها برای کاربر ویژه (و مدیران) و امکان ویرایش اطلاعات کانال‌ها توسط مدیران.
            \item \textbf{جستجو: } 
                این زیرسیستم، امکان جستجوی کانال‌های جدید برای عضویت را فراهم می‌کند.
        \end{itemize}
    
    \titr{سند نیازمندی‌ها}
        \zirtitr{فهرست اولویت‌بندی‌شده‌ی نیازمندی‌های وظیفه‌ای}
\begin{table}[H]
    \centering
    \begin{tabular}{|p{5cm}|c|p{4cm}|p{3cm}|}
        \hline
        نام نیازمندی  & اولویت & زیرسیستم مربوطه & ملاحظات\\\hline
        ثبت‌نام و ورود & \lr{Must Have} & ناحیه‌ی کاربری & \\\hline
        ثبت نام هم با ایمیل یا شماره تلفن & \lr{Could Have} & ناحیه‌ی کاربری & دست کم یکی باید باشد\\\hline
        ورود هم با ایمیل یا شماره تلفن & \lr{Could Have} & ناحیه‌ی کاربری & دست کم یکی باید باشد\\\hline
        ایجاد کانال & \lr{Must Have} & مدیریت کانال & \\\hline
        نهادن محتوا در کانال & \lr{Must Have} & مدیریت کانال & \\\hline
        انواع محتوای چندرسانه‌ای & \lr{Should Have} & مدیریت کانال & \\\hline
        انتساب یکی از نقش‌های چهارگانه به هر عضو کانال & \lr{Must Have} & مدیریت کانال & \\\hline
        امکان تعریف تعرفه‌های اشتراک & \lr{Must Have} & مدیریت اشتراک و پرداخت  & \\\hline
        انواع تعرفه (یک‌ماهه، سه‌ماهه، شش‌ماهه و یک‌ساله) & \lr{Could Have} & مدیریت اشتراک و پرداخت & \\\hline
        امکان تعیین مدیر توسط مالک & \lr{Should Have} & مدیریت کانال & \\\hline
        امکان عضو شدن در یک کانال  & \lr{Must Have} & ناحیه‌ی کاربری / مدیریت کانال & \\\hline
        امکان خریدن اشتراک & \lr{Must Have} & مدیریت اشتراک و پرداخت & \\\hline
        پرداختن مبالغ گردآوری شده‌ی کانال به مالک/مدیر(ها) & \lr{Must Have} & مدیریت اشتراک و پرداخت & \\\hline
        امکان تعیین سهم مدیر توسط مالک & \lr{Should Have} & مدیریت اشتراک و پرداخت & در صورت پیاده‌سازی امکان تعیین مدیر توسط مالک کانال \\\hline
        امکان مشاهده‌ی محتوای رایگان توسط همه‌ی اعضای کانال & \lr{Must Have} & مدیریت دسترسی و نمایش محتوا & \\\hline
        امکان مشاهده‌ی عنوان محتوای پولی برای اعضای عادی & \lr{Must Have} & مدیریت دسترسی و نمایش محتوا & \\\hline
        امکان نوشتن خلاصه برای محتوای پولی & \lr{Could Have} & مدیریت کانال & \\\hline
        امکان مشاهده‌ی خلاصه‌ی محتوای پولی برای اعضای عادی در صورت وجود خلاصه
            & \lr{Must Have} & مدیریت دسترسی و نمایش محتوا & در صورت پیاده‌سازی امکان نوشتن خلاصه برای محتوای پولی\\\hline
        امکان مشاهده‌ی محتوای پولی برای اعضای ویژه و مدیران و مالک & \lr{Must Have} & مدیریت دسترسی و نمایش محتوا & \\\hline
        امکان پرداختن موردی برای مشاهده‌ی یک محتوای پولی خاص بی خریدن اشتراک & \lr{Could Have} & 
            مدیریت اشتراک و پرداخت / مدیریت دسترسی و نمایش محتوا & \\\hline
        کیف پول و اعتبار حساب & \lr{Won't Have} & مدیریت اشتراک و پرداخت & \\\hline
        امکان مشاهده‌ی فهرست کانال‌هایی که در آن عضویم & \lr{Must Have} & ناحیه‌ی کاربری & \\\hline
        امکان جستجوی کانال‌های جدید برای عضویت & \lr{Must Have} & جستجو & \\\hline
    \end{tabular}
    \caption[اولویت‌بندی نیازمندی‌های وظیفه‌ای]{فهرست اولویت‌بندی‌شده‌ی نیازمندی‌های وظیفه‌ای با اولویت‌بندی \lr{MoSCoW}}
    \label{tab:functional_requirements_moscow}
\end{table}

\zirtitr{فهرست نیازمندی‌های غیروظیفه‌ای}
\begin{enumerate}[]
    \item امنیت اطلاعات خصوصی کاربران
    \item عدم امکان نفوذ و مشاهده‌ی محتوای پولی بی خریدن اشتراک
    \item نگه‌داری حساب مربوط به هر کانال
    \item اشتراک‌های خریده‌شده به درستی مدیریت شوند. این طور نشود که یک نفر اشتراک خریده‌باشد ولی نتواند ببیند یا اشتراکش تمام شده باشد ولی همچنان ببیند.
    \item امکان پاسخ‌گویی همزمان به شمار زیادی از کاربران در زمان مناسب
    \item فضای کافی سرور برای نگهداری داده‌ها
    \item کارایی سامانه
    \item ایمن بودن داده‌ها یعنی نپریدن اطلاعات کاربران که ممکن است باعث ضرر مالی شود.
    \item قابلیت توسعه و ویرایش
    \item قابلیت همکاری، یعنی سيستم بايد قابليت همکاری با سامانه‌ها و سرويس‌هاي ديگر را داشته باشد تا بتواند اطلاعات و منابع را با آن‌ها به اشتراک بگذارد يا از آن‌ها استفاده کند.
\end{enumerate}

\zirtitr{\lr {ArchitecturallySignificantRequirements}}


نیازمندی های با اولویت بالا و ریسک بالا را می توان در این دسته از نیازمندی ها در نظر گرفت. نیازمندی های رایج‌ که در این دسته قرار می‌ گیرند، مربوط به تکنولوژی های مورد استفاده در پروژه است.نیازمندی های تعیین کننده در معماری سیستم و ایجاد کننده محدودیت های تکنولوژی که ما شناسایی کردیم به شرح زیرند:
\begin{enumerate}[]
\item ۱ دوام و سازگاری داده ها

\item ۲امنیت داده ها 

\item ۳رابط کاربری آسان
\end{enumerate}

 نیازمندیهایی که از نظر مشتری اهمیت بیشتری داشتند به ترتیب اولویت:

 \begin{enumerate}[]
    
 \item   استفاده از زبان برنامه نویسی پایتون

 \item  استفاده از  django orm برای ارتباط با پایگاه داده

 \item  استفاده از پایگاه داده sqlite
 
\end{enumerate}


        
    \titr{سند ریسک‌ها}
        ریسک‌ها را به این صورت دسته‌بندی  می‌کنیم:\\
\large \textbf{ریسک‌های امنیتی}
\begin{enumerate}[start = 1]
    \item \textbf{نقض حریم خصوصی کاربران}
       \\\textbf{توضیح: }
        ممکن است داده‌های شخصی کاربران نظیر اطلاعات کاربری، رایانامه یا شماره‌ی تلفن یا  … در دسترس افراد غیرمجاز (همچون هکرها) قرار بگیرد. همچنین ممکن است داده‌ها هنگام جابه‌جایی میان سرور و کلاینت، در صورت عدم استفاده از پروتکل‌های امنیتی مناسب، در اختیار هکرها قرار بگیرد.
       \\\textbf{راهکار: }
        با رمزنگاری داده‌های حساس و استفاده از پروتکل‌های  امنیتی مناسب می‌توان با این خطر مقابله کرد.
    \item \textbf{خطر از میان رفتن تمام یا بخشی از داده‌ها}
       \\\textbf{توضیح: }
        ممکن است بنابر هر دلیلی، به سرور آسیبی برسد و در نتیجه، تمام یا بخشی از داده‌های کاربران از دست برود. این داده‌ها ممکن است در شرایطی باعث ضرر مالی شود برای نمونه اگر داده‌های مربوط به ثبت اشتراک‌‌های  یک کاربر از میان برود، شخص علی‌رغم این که پیشتر پول پرداخته،‌ نمی‌تواند از اشتراک خود استفاده کند که این یک نمونه‌ای است که باعث ضرر مالی می‌شود. یا ممکن است اطلاعات یک کانال که تعداد زیادی عضو دارد بپرد و روشن است که این باعث ضرر مالی بزرگی می‌تواند بشود.
       \\\textbf{راهکار: }
        بهتر است در بازه‌های زمانی مناسبی، از همه‌ی داده‌های روی سرور، پشتیبان بگیریم.
    \item \textbf{دسترسی غیر مجاز به محتوای پولی}
       \\\textbf{توضیح: }
        ممکن است با روش‌هایی ناشی از نقص‌های امنیتی، برخی افراد غیرمجاز بتوانند بی آنکه اشتراک بخرند، به محتواهای پولی کانال‌ها دسترسی بیابند. این اتّفاق می‌تواند باعث نارضایتی تولیدکنندگان محتوا و در نتیجه از میان رفتن شبکه‌ی اجتماعی قاصدک شود.
       \\\textbf{راهکار: }
        باید همه‌ی حفره‌های  امنیتی سامانه را به دقّت شناخت و با روش‌های مناسب مانند به‌کارگیری پروتکل‌های امنیتی مناسب، بررسی ایمنی سرورها در برابر نرم‌افزارهای مخرّب و … و انجام آزمون‌های نرم‌افزاری لازم، جلوی این موضوع را گرفت.
    \item \textbf{دسترسی  افراد غیرمجاز به اطلاعات کانال‌ها}
       \\\textbf{توضیح: }
        ممکن است افرادی که مدیر کانال نیستند بتوانند با روش‌هایی ناشی از نقص امنیتی، به اطلاعات کانال‌ها دسترسی بیابند و در کانال‌ها پست بگذارند یا تغییراتی در کانال ایجاد کنند که به روشنی نامطلوب است. همچنین ممکن است برخی مدیران که مالک نیستند بتوانند با دسترسی به امکانات مالک، فعالیت‌های غیر مجازی مانند افزایش سهم خود انجام دهند.
       \\\textbf{راهکار: }
        باید همه‌ی حفره‌های  امنیتی سامانه را به دقّت شناخت و با روش‌های مناسب مانند به‌کارگیری پروتکل‌های امنیتی مناسب، بررسی ایمنی سرورها در برابر نرم‌افزارهای مخرّب و … و انجام آزمون‌های نرم‌افزاری لازم، جلوی این موضوع را گرفت.
        \item \textbf{امکان بارگذاری محتوای مخاطره‌آمیز مانند فایل‌های ویروسی و مخرّب}
       \\\textbf{توضیح: }
        ممکن است هنگام بارگذاری فایل‌های چندرسانه‌ای مانند ویدیو، فایل مورد نظر مخرّب یا ویروسی باشد و کاربرانی که آن را دریافت می‌کنند یا حتّی سرور که فایل روی آن قرار می‌گیرد، به خطر بیفتد.
       \\\textbf{راهکار: }
        برای جلوگیری از این موضوع می‌توان آنتی‌ویروس خوبی روی سرور نصب کرد تا جلوی فایل‌های مخرّب را بگیرد و با تشخیص آنها، آنها را حذف کند. همچنین در راستای ایمنی بیشتر، می‌توان در کلاینت نیز این موضوع را بررسی اوّلیّه‌ای کرد تا اگر فایل مخرّب بود، از فرستادن آن به سرور جلوگیریم.
    \item \textbf{خطا در انتساب نقش‌های چهارگانه}
       \\\textbf{توضیح: }
        ممکن است سامانه در انتساب نقش‌های چهارگانه (مالک، مدیر، کاربر ویژه، کاربر عادّی) دچار اشتباه شود و در نتیجه دسترسی‌های نادرستی به افراد داده شود. برای مثال به کاربری که اشتراک دارد به جای «کاربر ویژه»، «کاربر عادّی» نسبت داده شود و در نتیجه دسترسی به محتوای پولی نداشته باشد که این به روشنی نامطلوب است.
       \\\textbf{راهکار: }
        با انجام آزمون‌های کیفی مورد نیاز، باید از صحّت عملکرد سامانه در سناریوهای گوناگون مطمئن شد. همچنین باید امنیت سیستم تأمین شود.
    \item \textbf{خطا در محاسبه‌ی تعرفه‌ها}
       \\\textbf{توضیح: }
        ممکن است هنگام نمایش تعرفه‌ها و پرداخت آنها، مقدار اشتباهی پول از کاربر گرفته شود. این مشکل ممکن است به علّت ایراد در سامانه باشد یا به خاطر خرابکاری عامدانه برای کاستن مقدار پول پرداخت شده باشد.
       \\\textbf{راهکار: }
        همچنان راهکارمان برای مقابله با این خطر، انجام آزمون‌های کیفی مناسب و مقابله با نواقص امنیتی و نفوذپذیری سامانه است.
    \item \textbf{خطا در محاسبه‌ی مدت اشتراک}
       \\\textbf{توضیح: }
        ممکن است مدّت اشتراک کاربران ویژه اشتباه محاسبه شود برای نمونه اگر اشتراک کاربری کمتر از مقدار واقعی محاسبه شود و در نتیجه زودتر از به پایان رسیدن آن، جلوی کاربر از دیدن محتویات پولی، گرفته شود یا اشتراک بیشتر از مدّت واقعی محاسبه شود و پس از پایان آن همچنان اجازه‌ی مشاهده‌ی محتویات پولی داده شود. این موضوع ممکن است به دلیل باگ در سامانه یا نواقص امنیتی و سوء استفاده‌های عامدانه انجام شود.
       \\\textbf{راهکار: }
        همچنان راهکارمان برای مقابله با این خطر، انجام آزمون‌های کیفی مناسب و مقابله با نواقص امنیتی و نفوذپذیری سامانه است.
\end{enumerate}

\large \textbf{ریسک‌های مربوط به محدوده و نیازمندی‌ها}
\begin{enumerate}[start = 9]       
    \item \textbf{روشن نبودن خواسته‌های مشتری یا درک نادرست نیازمندی‌ها توسط تیم ایجاد}
       \\\textbf{توضیح: }
        ممکن است برخی خواسته‌های مشتری به طور کامل روشن نباشد یا توسط تیم ایجاد به درستی فهمیده نشود. این ممکن است به این خاطر باشد که انتقال خواسته‌های مشتری به تیم ایجاد به درستی صورت نگرفته یا آن که برخی از خواسته‌های مشتری، برای خودش هم به طور کامل مشخّص نبوده و با جلوتر رفتن برایش روشن می‌شود.
       \\\textbf{راهکار: }
        پیش از آغاز پروژه، نیازمندی‌ها تحلیل شوند و به تأیید مشتری برسند. همچنین باید مدام با مشتری ارتباط داشت تا جلوی بروز مشکلات را گرفت یا مشکلات از این دست را هرچه زودتر تشخیص داده و حل کرد. نیز باید قابلیت اصلاح‌پذیری و maintenance پروژه بالا باشد تا بتوان در صورت بروز نیازمندی‌های جدید یا تغییر خواسته‌های مشتری، اصلاحات لازمه را به سادگی پیاده ساخت.
    \item \textbf{بزرگ‌شدن بیش از حد پروژه}
       \\\textbf{توضیح: }
        ا جلو رفتن پروژه ممکن است نیازمندی‌های جدید بروز یابند و خواسته‌های مشتری بیشتر شود و این ممکن است به تدریج باعث شود پروژه بیش از اندازه بزرگ شود و پیاده‌سازی آن برای تیم ایجاد، با شرایط و امکانات موجود مشکل یا ناممکن شود.
       \\\textbf{راهکار: }
        برای مقابله با این ریسک، راهکارهایی در نظر گرفته‌ایم از جمله تعریف دقیق نیازمندی‌ها، توافق با مشتری بر سقف قیمت و زمان و اندازه.
    \end{enumerate}

\large \textbf{ریسک‌های مربوط به زمان‌بندی}
\begin{enumerate}[start = 11] 
    \item \textbf{تخمین‌های زمانی نادرست برای انجام تسک‌ها}
       \\\textbf{توضیح: }
        ممکن است برخی  از کارها در زمان پیش‌بینی شده تمام نشوند. در اینجا اتفاقات پیش‌بینی نشده مد نظر نیستند و مسئله تخمین نادرست از زمان مورد نیاز برای انجام کارها است.
       \\\textbf{راهکار: }
        باید هنگام برنامه‌ریزی، کوشید که تا جای ممکن با دقّت بالایی زمان‌بندی پروژه را پیش‌بینی کرد و برنامه‌ریزی دقیقی داشت و در این کار باید از اشخاص با تجربه کمک گرفت. با این وجود همچنان این مشکل ممکن است که باید در آن صورت بتوان برنامه‌ریزی را طوری تغییر داد که کمترین آسیب به روند پروژه وارد آید و پروژه شکست نخورد. همچنین می‌توان هنگام اعلام ددلاین به مشتری، مقداری ددلاین را بیشتر از زمان پیش‌بینی‌شده اعلام کرد تا در صورت پیش‌آمدن این مشکل، به مشتری نگرانی‌ای تحمیل نشود.
    \item \textbf{اعلام دیرهنگام بسترهای مجاز برای انجام پروژه}
       \\\textbf{توضیح: }
        این مورد ممکن است که به تاخیر در یادگیری یک بستر جدید منجر شود که  
        در نهایت به تاخیر در تحویل پروژه بیانجامد.    
       \\\textbf{راهکار: }
        پیدا کردن روشی برای انتقال حداکثری موارد تولید شده به بستر جدید،استفاده از بستری مجازی  نزدیک به \lr{contex} پروژه و   بستر هایی که‌قبلا برای پروژه های مشابه استفاده شده
    \item \textbf{ناتوانی در تحویل فازهای پروژه در ددلاین مشخص شده}
       \\\textbf{توضیح: }
        به دلایل مختلف مانند کوتاهی اعضای گروه در انجام وظایف تقسیم بندی شده فاز های پروژه در ددلاین مشخص شده تحویل داده نمی شود
       \\\textbf{راهکار: }
        شناسایی  مهارت های های اعضا و محول کردن تسک هایی در جهت مهارت ها 
\end{enumerate}

\large \textbf{ریسک‌های مربوط به مدیریت تغییرات}
\begin{enumerate}[start = 14] 

    \item \textbf{نبود قابلیت Maintainability }
       \\\textbf{توضیح: }
        در صورت نبود Maintability شاهد افت کیفیت خواهیم بود
        \\\textbf{راهکار: }
        اصول ظراحی شی گرا را رعایت کنیم و پیاده سازی خوبی داشته باشیم
    \end{enumerate}

\large \textbf{ریسک‌های مربوط به ارتباطات}
\begin{enumerate}[start = 15] 
    \item \textbf{کمرنگ شدن ارتباط دستیاران با تیم}
       \\\textbf{توضیح: }
        فعالیت دستیاران درس در طول ترم کمرنگ شود و بازخوردها به موقع ارائه نشوند.
       \\\textbf{راهکار: }
        پیگیری از دستیاران با ایمیل و راه های ارتباطی دیگر
\end{enumerate}

\large \textbf{ریسک‌های مربوط به نیروی انسانی}
\begin{enumerate}[start = 16] 
        \item \textbf{خروج عضوی از تیم پیش از پایان یافتن پروژه            }
       \\\textbf{توضیح: }
        یکی از اعضای تیم، پیش از پایان یافتن پروژه ممکن است جهت تحصیل در خارج از کشور مهاجرت کند و از تیم خارج شود.
       \\\textbf{راهکار: }
        سه راه حل برای این ریسک قابل تصور است که به اقتضای موقعیت، یک باید انتخاب شود: 
        ١. انجام تسک ها به صورت فشرده ای باشد تا پیش از مهاجرت عضو، پروژه پایان یافته باشد. 
        ٢. در حین انجام پروژه، بقیه اعضای تیم در جریان جزییات تسک هایی که آن فرد انجام مͬ دهد قرار گیرند تا بتوانند در صورت لزوم ادامه ی تسک های وی را انجام دهند.
         ٣. امکان برقراری ارتباط از راه دور با عضو خارج شده میسر باشد تا همکاری بدین صورت انجام شود.
        
    \item \textbf{پیش‌آمدن مشکل‌های غیر مترقبه}
       \\\textbf{توضیح: }
        برای نمونه، امتحان های درسی ممکن است منجر به در دسترس نبودن یک یا چند عضو در یک بازه‌ی زمانی بحرانی شود.
       \\\textbf{راهکار: }
        در هنگام تخمین های زمانی، باید این موارد تا حد امکان در نظر گرفته شوند.
    \item \textbf{پیش‌آمدن مشکل برای اعضای تیم در زمان‌های خاص}
       \\\textbf{توضیح: }
        پروژه های درس های دیگر ممکن است باعث شود که برنامه‌ریزی‌ها با مشکل مواجه شوند که خود می تواند منجر به کاهش کیفیت محصولات ترخیص شود.
       \\\textbf{راهکار: }
        برنامه ریزی مناسب و مدیریت اعضای تیم
\end{enumerate}

\large \textbf{ریسک‌های فنّی}
\begin{enumerate}[start = 19] 
        \item \textbf{بروز مشکلات سخت‌افزاری و نرم‌افزاری}
       \\\textbf{توضیح: }
        برای نمونه خراب شدن کامپیوتر/لپ‌تاپ یکی از اعضای تیم، بروز ایراد طولانی مدت در اتصال اینترنت یا بروز اشکال در سرور سایت Trello یا Github 
       \\\textbf{راهکار: }
        برای مقابله با مشکلات نرم افزاری مربوط به شبکه، لازم است سعی شود همواره یک رونوشت از آخرین محصولات به صورت offline موجود باشد.
    \item \textbf{نیاز به یادگیری مهارت‌های جدید}
       \\\textbf{توضیح: }
        ممکن است لازم شود اعضای تیم موارد جدیدی را یاد بگیرند که باعث نادقیق بودن تخمین های زمانی شود
       \\\textbf{راهکار: }
        می‌توان کوشید تا جای ممکن از روش‌هایی استفاده‌کرد که اعضای تیم مهارتش را دارند و در صورت نیاز از آموزش‌های مناسب و کمک‌گرفتن از افرادی که آن مهارت‌ها را بلد هستند بهره جست.
    \item \textbf{نامناسب‌بودن ابزارهای CASE مورد استفاده}
       \\\textbf{توضیح: }
        ممکن است ابزارهای CASE مورد استفاده، بعضی از نمودارهایی را که لازم است تولید شوند پشتیبانی نکند. در نتیجه هماهنگ کردن نمودارهای مختلف با یکدیگر ممکن است مشکل‌ساز شود. همچنین، ممکن است اعضای تیم ایجاد شناخت دقیقی از ابزار CASE مورد استفاده نداشته باشند و انتظاری داشته باشند که با این ابزار قابل رفع نباشد.
       \\\textbf{راهکار: }
        تشویق اعضای تیم به مطالعه‌ی دقیق مستندات و راهنماهای ابزارهای مختلف CASE پیش از انتخاب و استفاده تا حد خوبی این ریسک را کاهش می‌دهد.
    \item \textbf{عدم رضایت مشتری از محصول نهایی}
       \\\textbf{توضیح: }
        در صورتی که محصول نهایی در آزمون پذیرش رد شود، هر چقدر هم که از استانداردهای کیفی تبعیت کرده باشد، پروژه شکست خورده اعلام خواهد
       \\\textbf{راهکار: }
        با تحلیل دقیق، تعامل بیشتر با مشتری و اعمال بازخوردهای هر فاز، می توان ریسک مورد قبول واقع نشدن محصول نهایی را کاهش داد.
    \item \textbf{مشکل در سامانه‌ی پرداختن پول}
       \\\textbf{توضیح: }
        مشکلات در زمینه پرداخت مثل از کار افتادن درگاه‌های بانکی
       \\\textbf{راهکار: }
        راه جایگزین به جای پرداخت پول آنلاین مثل پرداخت حضوری در عابر بانک‌ها و هماهنگی با پشتیبانی، یا صبر کردن تا درست‌شدن یا استفاده از رمزارزها
    \item \textbf{ناکافی بودن فضای ذخیره‌سازی سرور برای نگه‌داری داده‌های سیستم}
       \\\textbf{توضیح: }
        فضای ذخیره‌سازی درنظرگرفته‌شده برای نگه‌داری داده‌های مورد نیاز کافی نیست.
       \\\textbf{راهکار: }
        افزودن  سرورهای ذخیره‌سازی با در نظر گیری بودجه
    \item \textbf{مقیاس‌پذیری و توانایی پاسخ‌گویی به شمار زیادی کاربر به طور همزمان به‌شیوه‌ای کارا        }
       \\\textbf{توضیح: }
        در بعضی موارد لازم  تعداد زیادی از کاربران به صورت همزمان خدمت داده شوند و ترافیک و بار سیستم به شدت بالا می‌رود و لازم است تدابیری برای آن اندیشیده شود.
       \\\textbf{راهکار: }
        داشتن سرور پشتیبان
    \item \textbf{تأمین‌ هزینه‌ی سرورها و \dots}
       \\\textbf{توضیح: }
        برای پیاده سازی سامانه احتیاج به سرور قدرتمند داریم
       \\\textbf{راهکار: }
        خرید سرورها میتواند از سمت مشتری به‌جای تیم ایجاد انجام شود. همچنین می‌توان برای پروژه اسپانسر یافت.
    \item \textbf{پرداختن حقوق کارکنان}
       \\\textbf{توضیح: }
        در صورت دیرکرد در پرداخت حقوق کارکنان ممکن است پروژه پیش نرود و به تاخیر بیافتد 
       \\\textbf{راهکار: }
        گرفتن اسپانسر یا دادن فیش حقوقی به کارمندان
\end{enumerate}

\zirtitr{اولویت‌بندی ریسک‌ها}
        در این جا ریسک‌ها را بر اساس اولویت در ۷ دسته قرار داده‌ایم
        امّا برای ریسک‌های در یک دسته‌ی اولویتی، از نظر اولویت
        تفاوتی قائل نشده‌ایم. در این اولویت‌بندی، کوشیده‌ایم
        به معیارهای نقض حریم خصوصی و مشکلات امنیتی و مسائل مالی
        اهمّیّت بالاتری بدهیم و نیز با توجّه به شدّت و احتمال موارد
        و سختی حل کردن آنها، مرتّب کنیم.

        \zirzirtitr{اولویت بسیار بالا}
        \begin{itemize}
            \item نقض حریم خصوصی کاربران
            \item دسترسی غیرمجاز به محتوای پولی
            \item خطاهای پرداخت
            \item خطر از میان رفتن تمام یا بخشی از داده‌ها
        \end{itemize}

        \zirzirtitr{اولویت بالا}
        \begin{itemize}
            \item دسترسی افراد غیرمجاز به اطلاعات کانال‌ها
            \item امکان بارگذاری محتوای مخاطره‌آمیز مانند فایل‌های ویروسی و مخرّب
            \item خطا در انتساب نقش‌های چهارگانه
            \item خطا در محاسبه‌ی تعرفه‌ها
            \item خطا در محاسبه‌ی مدت اشتراک
        \end{itemize}

        \zirzirtitr{اولویت نسبتاً بالا}
        \begin{itemize}
            \item روشن‌نبودن محدوده‌ی پروژه و درک نادرست نیازمندی‌ها توسط تیم ایجاد
            \item عدم رضایت مشتری از محصول نهایی
        \end{itemize}

        \zirzirtitr{اولویت متوسط}
        \begin{itemize}
            \item نبود قابلیت Maintability ودر نتیجه افت کیفیت
            \item ناکافی بودن فضای ذخیره‌سازی سرور برای نگه‌داری داده‌های سیستم
            \item نیاز به یادگیری مهارت‌های جدید
            \item مقیاس‌پذیری و توانایی پاسخ‌گویی به شمار زیادی کاربر به طور همزمان به‌شیوه‌ای کارا
        \end{itemize}

        \zirzirtitr{اولویت نسبتاً پایین}
        \begin{itemize}
            \item خروج عضوی از تیم پیش از پایان یافتن پروژه
            \item پیش‌آمدن مشکل‌های غیر مترقبه
            \item پیش‌آمدن مشکل برای اعضای تیم در زمان‌های خاص
        \end{itemize}

        \zirzirtitr{اولویت پایین}
        \begin{itemize}
            \item بروز مشکلات سخت‌افزاری و نرم‌افزاری
            \item نامناسب بودن ابزارهای CASE مورد استفاده
            \item مشکل در سامانه‌ی پرداختن پول (مثلاً از کار افتادن درگاه بانکی)
        \end{itemize}

        \zirzirtitr{اولویت پایین‌تر}
        \begin{itemize}
            \item تأمین‌ هزینه‌ی سرورها و \dots
            \item پرداختن حقوق کارکنان
            \item بزرگ‌شدن بیش از حد پروژه
            \item تخمین‌های زمانی نادرست برای انجام تسک‌ها
            \item اعلام دیرهنگام بسترهای مجاز برای انجام پروژه
            \item ناتوانی در تحویل فازهای پروژه در ددلاین مشخص شده
            \item ریسک‌های مربوط به ارتباطات
            \item کمرنگ شدن ارتباط دستیاران با تیم
        \end{itemize}
            
        \titr{ریسک های تکنیکی}
        
\begin{itemize}
\item \textbf{بروز مشکلات سخت افزاری و نرم افزاری}\\
\textbf{توضیح} برای نمونه خراب شدن کامپیوتر/لپتاپ یکی از اعضای تیم، بروز ایراد طولانی مدت در اتصال اینترنت یا بروز اشکال طولانی مدت برای Microsoft to do یا Telegram یا .Github در صورتی که هر یک از این اتفاقات رخ دهد، زمانبندی ها با مشکل مواجه می شود. در مورد بروز مشکلات در سه مورد آخر، ارتباط از راه دور نیز برای اعضای تیم دشوار می گردد.\\
\textbf{راه حل} لازم است سعی شود همواره یک رونوشت از آخرین محصولات به صورت برون خط موجود باشد تا در صورت آسیب کامپیوتر اعضا یا عدم امکان برقراری ارتباط با سایت ،Github کماکان بتوان روی کدها کار کرد.

از آنجایی که که نگهداری مستندات و تقسیم وظایف از طریق Microsoft to doانجام می گردد، در صورتی که تقسیم کار و مستندات را به صورت برونخط داشته باشیم، در صورت بروز ایراد درMicrosoft to do مشکلا جدی برای تیم رخ نخواهد داد.در صورتی که Telegram با مشکلات قطعی مواجه شود، می توان از خدمات جایگزین، مانند ایمیل استفاده کرد. 

\item \textbf{نیاز به یادگیری مهارت های جدید}\\
\textbf{توضیح} ممکن است لازم شود اعضای تیم موارد جدیدی را یاد بگیرند که باعث نادقیق بودن تخمین های زمانی شود\\
\textbf{راه حل}  یادگیری مهارت های جدید از طريق منابع مناسب نظير ويديوهای موجود در يوتيوب و منابع آموزشي و استفاده از تجارب افراد متخصص در ان زمينه كه باعث تسريع يادگيري استفاده از آن تكنولوژي مي شود. 

\item \textbf{نامناسب بودن ابزارهای CASE مورد استفاده}\\
\textbf{توضيح} ممكن است ابزارهای CASE مورد استفاده، بعضي از نمودارهایي را كه لازم است توليد شوند پشتيبانی نكند. در نتيجه هماهنگ كردن نمودارهای مختلف با يكديگر ممكن است مشكل ساز شود. همچنين، ممكن است اعضای تيم ايجاد شناخت دقيقی از ابزار CASE مورد استفاده نداشته باشند و انتظاري داشته باشند كه با اين ابزار قابل رفع نباشد.\\
\textbf{راه حل} تشویق اعضای تیم به مطالعه ی دقیق مستندات و راهنماهای ابزارهای مختلف CASE پیش از انتخاب و استفاده تا حد خوبی این ریسک را کاهش می دهد. 

\item \textbf{دشواری مربوط به پایگاه داده}\\
\textbf{توضیح} ممکن است در هنگام پیاده سازی، برای برقراری ارتباط با پایگاه داده مشکلاتی از لحاظ فنی رخ بدهد که به علت عدم تحربه ی کافی اعضا در این زمینه، محتمل است. در صورت وقوع، تخمین های زمان􏰀 با مشکل مو\\
\textbf{راه حل}  استفاده از منابع آموزشی مناسب برای یادگیری نحوه‌ی برقراری ارتباط با پایگاه داده و کمک‌گرفتن از تجارب و راهنمایی‌های دستیاران درس و دانشجویان دیگر.
\end{itemize} 
   
    \titr{موارد کاربرد}
        \zirtitr{توضیحات هر مورد کاربرد}
            \newcounter{usecaseid}
            \setcounter{usecaseid}{0}

            \newcommand{\usecase}[9]{
                \stepcounter{usecaseid}
                \begin{table}
                    \centering
                    \begin{tabular}{|c|p{7cm}|}
                        \hline
                        شناسه & \arabic{usecaseid}\\\hline
                        نام & #1\\\hline
                        شرح & #2\\\hline
                        زیرسیستم & #3\\\hline
                        کنشگر(های) اوّلیه & #4\\\hline
                        کنشگر(های) ثانویه & #5\\\hline
                        جریان اصلی & #6\\\hline
                        پیش‌نیاز(ها) & #7\\\hline
                        پس‌نیاز(ها) & #8\\\hline
                        جریان جایگزین & #9\\\hline
    
                    \end{tabular}
                \end{table}
            }
            \usecase{ثبت‌نام کاربر}{
                    به کاربران امکان می‌دهد تا یک حساب کاربری جدید در شبکه‌ی اجتماعی قاصدک ایجاد کنند
                }{ناحیه‌ی کاربری}{کاربران}{-}{
                    \begin{enumerate}
                        \item کاربر درخواست ثبت نام را ارسال می کند
                        \item کاربر اطلاعات لازم (نام ، نام کاربری،ایمیل،گذرواژه و تکرار)  را وارد میکند
                        \item سیستم اطلاعات را بررسی میکند
                            \begin{itemize}
                                \item[3.1] بررسی معتبر بودن فرمت ایمیل
                                \item[3.2] بررسی تکراری نبودن ایمیل
                                \item[3.3] بررسی یکسان بودن گذرواژه و تکرار آن
                            \end{itemize}
                        \item سیستم یک حساب کاربری جدید برای کاربر ایجاد میکند
                    \end{enumerate}
                }{-}{حساب کاربری با موفقیت ایجاد شود.}{تکراری بودن ایمیل یا شماره تلفن}

            \usecase{ورود کاربر}{به کاربران امکان می دهد تا با استفاده از ایمیل یا شماره تلفن خود وارد حساب کاربری خود شوند}{
                    ناحیه کاربری
                }{کاربران}{}{
                    \begin{enumerate}
                        \item کاربر درخواست ورود به سامانه را ارسال میکند
                        \item کاربر مشخصات کاربری خواسته شده (نام کاربری،گذرواژه) را  وارد میکند
                        \item سیستم مشخصات کاربر را بررسی میکند
                        \item اگر مشخصات به درستی وارد شده باشند، کاربر را به صفحه مربوطه هدایت میکند.
                        \item اگر مشخصات اشتباه و یا ناقص وارد شده باشند، سیستم یک پیغام خطا به کاربر نشان می دهد و وی را به صفحه ورود باز میگرداند
                    \end{enumerate}
                }{حساب کاربری قبلا ایجاد شده باشد}{کاربر با موفقیت وارد حساب کاربری شود}{غلط بودن اطلاعات احراز هویت کاربر
                
                پیغام خطا مبنی بر اهراز هویت ناموفق
                }

            \usecase{ایجاد کانال}{یک کانال جدید ایجاد شود }{مدیریت کانال}{کاربران}{-}{
                    \begin{enumerate}
                        \item کاربر وارد بخش مربوط به ایجاد کانال میشود
                        \item کاربر اطلاعات لازم (نام کانال ، موضوع کانال) برای ایجاد کانال را وارد میکند
                        \item سیستم اطلاعات را بررسی میکند
                        \item اگر اطلاعات صحیح بود سیستم یک کانال جدید را ایجاد میکند
                        \item وگرنه پیام خطای مرتبط به کاربر نمایش داده میشود

                    \end{enumerate}
                }{کاربر با موفقیت وارد حساب کاربری خود شود}{کانال با موفقیت ایجاد شود}{انصراف}

            \usecase{افزودن محتوا به کانال}{
                    به کاربران امکان می دهد تا محتواهای جدید را به یک کانال اضافه کنند
                }{مدیریت کانال}{کاربران(مالک و مدیر کانال)(های) کانال}{-}{
                    \begin{enumerate}
                        \item کاربر وارد بخش مربوط به افزودن محتوا میشود
                        \item کاربر اطلاعات لازم (عنوان محتوا، خلاصه محتوا، محتوا، هزینه محتوا(در صورت پولی بودن)) را وارد میکند
                        \item سیستم اطلاعات را بررسی میکند
                        \item اگر اطلاعات صحیح بود سیستم محتوای جدید را به کانال اضافه میکند
                        \item وگرنه پیام خطای مرتبط به کاربر نمایش داده میشود

                    \end{enumerate}
                }{کاربر با موفقیت وارد حساب کاربری شود

                کانال مدنظر ایجاد شده باشد}{محتوا با موفقیت به کانال اضافه شود}{انصراف}

            \usecase{تعریف تعرفه های اشتراک}{
                به مالک کانال امکان میدهد تا تعرفه های مختلفی برای اشتراک در کانال تعریف کنند
                }{مدیریت اشتراک و پرداخت}{مالک کانال}{-}{
                    \begin{enumerate}
                        \item مالک کانال وارد بخش مربوط به تعریف تعرفه های کانال می‌شود 
                        \item مدیر کانال تعرفه های مختلفی برای اشتراک در کانال تعریف می‌کند
                        \item سیستم تعرفه ها را در کانال ذخیره می‌کند
                    \end{enumerate}
                }{کاربر با موفقیت وارد حساب کاربری شده باشد

                کانال ایجاد شده باشد

                نقش کاربر،مالک کانال باشد}{تعرفه های اشتراک در کانال تعریف شود}{-}

            \usecase{تعیین مدیر کانال}{
                    به مالک کانال این امکان را میدهد تا مدیر کانال را مشخص کند
                }{مدیریت کانال}{مالک کانال}{-}{
                    \begin{itemize}
                        \item مالک کانال وارد بخش مربوط به تعیین مدیر می شود
                        \item لیست کاربران برای مالک نمایش داده میشود
                        \item یکی از اعضای کانال را به عنوان مدیر آن کانال انتخاب میکند
                        \item مالک کانال سهم مدیر را در درآمد کانال تعیین میکند
                        \item سیستم مدیر کانال را ذخیره میکند

                    \end{itemize}
                }{
                    کاربر با موفقیت وارد حساب کاربری شده باشد

                    کانال ایجاد شده باشد

                    مالک کانال وجود داشته باشد
                }{مدیر کانال تعیین و ذخیره شود}{انصراف}

            \usecase{عضو شدن در کانال}{به کاربران امکان میدهد به یک کانال موجود پیوسته و عضو آن شوند.}{ناحیه کاربری}{کاربران}{-}{
                    \begin{enumerate}
                        \item کاربر درخواست پیوستن به آن کانال را می‌دهد

                        \item سیستم درخواست را بررسی کرده و کاربر را به عضو کانال اضافه می‌کند
                        \item وگرنه پیام خطای مرتبط را نمایش می دهد 
                        \item کانال به فهرست کانال‌های عضو شده‌ی کاربر اضافه می‌شود
                    \end{enumerate}
                }{کاربر با موفقیت وارد حساب کاربری شده باشد}{کاربر عضو کانال شود}{انصراف}



            \usecase{خرید اشتراک}{به کاربران امکان میدهد اشتراکی را در یک کانال خریداری کنند }{مدیریت اشتراک و پرداخت}{کاربران}{-}{
                \begin{enumerate}
                    \item کاربر وارد بخش مربوط به اشتراک ها می‌شود
                    \item کاربر کانال مورد نظر خود را جستجو میکند
                    \item کاربر اشتراک مورد نظر خود را انتخاب کرده و درخواست خرید را ارسال میکند
                    \item اگر عملیات پرداخت موفقیت آمیز نبود پیام خطای عملیاتی برای کاربر ارسال میشود در غیر اینصورت اشتراک را به کاربر اختصاص میدهد

                \end{enumerate}
                }{کاربر با موفقیت وارد حساب کاربری شده باشد.
                
                کاربر عضو کانال مورد نظر باشد
                
                کانال مورد نظر خود را بتواند جستجو کرده باشد.
                }{
                    اشتراک مورد نظر به کاربر اختصاص داده شود و نقش کاربر به کاربر ویژه تبدیل شود.
                }{انصراف}

            \usecase{تسویه مالی کانال}{
                    مبالغی که از اشتراک‌ها و محتواهای پولی دریافت می‌شود را به مالک و مدیر کانال بپردازد
                }{مدیریت اشتراک و پرداخت}{زمان

                مالک کانال

                مدیر کانال
                }{بانک}{
                    \begin{enumerate}
                        \item سیستم مبالغ دریافت شده از اشتراک ها و محتواهای پرداختی را محاسبه میکند

                        \item سیستم مبالغ را به مالک و مدیر کانال می‌پردازد
                        \item تصفیه‌ی حساب ثبت می‌شود
                    \end{enumerate}
                }{برای محتواهای پولی کانال، به صورت موردی یا اشتراکی، پولی پرداخته‌شده‌باشد که هنوز تصفیه حساب نشده باشد}{مبالغ به مالک و مدیر کانال پرداخت شود
                
                در سامانه ثبت‌شود که تصفیه‌ی حساب انجام شده.}{-}
            

            \usecase{تعیین سهم مالک و مدیر کانال }{
                    به مالک کانال‌ها امکان میدهد تا سهم مدیران را در درآمد حاصل از کانال خود، تعیین کنند
                }{مدیریت اشتراک و پرداخت}{مالک کانال}{-}{
                    \begin{enumerate}
                        \item مالک کانال سهم مدیر(ها) را در درآمد کانال تعیین میکند
                    \end{enumerate}
                }{کاربر با موفقیت وارد حساب کاربری شده
                
                کانال ایجاد شده باشد

                نقش مالک کانال را داشته باشد

                دست کم یک مدیر کانال وجود داشته باشد
                }{سهم مدیر و مالک کانال تعیین شود}{-}
                
                \usecase{مشاهده پست}{
                    به اعضای کانال از جمله مدیران اجازه‌ی مشاهده‌ی پست‌ها را با توجه به سطح دسترسی‌شان می‌دهد.
                }{مدیریت دسترسی و نمایش محتوا}{کاربران}{-}{
                    \begin{enumerate}
                        \item کاربر می خواهد پستی را ببیند
                        \item در صورتی که پست رایگان باشد آن را نمایش می دهد
                        \item وگرنه
                        \item ۱-۳- اگر کاربر، مالک، مدیر، یا عضو ویژه کانال باشد یا پست را به صورت موردی خریده باشد آن را نمایش می دهد
                        \item ۲-۳- وگرنه عنوان و خلاصه پست را نشان می دهد

                    \end{enumerate}
                }{کاربر با موفقیت وارد حساب کاربری شده
                
                حداقل عضو یک کانال باشد

                }{کاربر پست های رایگان کانال را مشاهده کند

                }{انصراف}

           
            
                \usecase{پرداخت موردی برای مشاهده مختوای پولی}{
                    به کاربران امکان می دهد برای مشاهده محتوای پرداختی یک کانال، مبلغ مورد نظر را پرداخت کنند
                }{مدیریت اشتراک و پرداخت ها}{کاربران}{
                    بانک
                }{
                    \begin{enumerate}
                        \item محتوای پولی خود را انتخاب میکند و درخواست خرید می‌دهد
                        \item برای پست هزینه‌ را می‌پردازد
                        \item در صورتی که پرداخت موفقیت آمیز بود
                        دسترسی به آن محتوای خاص به کاربر داده میشود
                        \item وگرنه پیام خطای مرتبط نمایش می دهد
                    \end{enumerate}
                }{کاربر با موفقیت وارد حساب کاربری شده
                
                حداقل یک کانال را دنبال کند

                 پست پیشتر توسط آن کاربر خریده نشده باشد.
                }
                { کاربر پرداخت را انجام داده و دسترسی به محتوای ویژه کانال را داشته باشد
                }{انصراف} 

            \usecase{نمایش فهرست کانال های عضو شده}{
                    به کاربران این امکان را می‌دهد لیست کانال‌هایی که عضویت داشته‌اند را مشاهده کنند.
                }{ناحیه کاربری}{کاربران}{
                    -
                }{
                    \begin{enumerate}
                        \item کاربر می‌خواهد فهرست کانال‌هایی را که در آنها عضو است ببیند
                        \item سیستم فهرست این کانال‌ها را به کاربر نشان می‌دهد

                    \end{enumerate}
                }{کاربر با موفقیت وارد حساب کاربری شده باشد}{
                    تمام کانال های کاربر برایش نمایش داده شود
                }{انصراف}

            \usecase{جستجو و مشاهده کانال‌های جدید}{
                    کاربر بتواند از طریق پنل جستجو کانال‌های جدیدی را برای عضویت پیدا کند.
                }{جستجو}{کاربران}{-}{
                    \begin{enumerate}
                        \item کاربر عنوان کانال مد نظر را در سیستم جستجو ، وارد می کند. 
                        \item لیستی از کانال‌ها نمایش داده می‌شود.
                        \item کاربر کانال مورد نظر را برای عضو شدن را انتخاب می کند
                    \end{enumerate}
                }{ورود به حساب کاربری}{کانال‌ها بر اساس ارتباط با کلمه جستجو شده مرتب می شوند}{انصراف}

                \usecase{خروج کاربر}{
                    کاربر بتواند از حساب کاربری خود خارج شود
                }{اهراز هویت}{کاربران}{-}{
                    \begin{enumerate}
                        \item   کاربر درخواست خروج از سامانه را ارسال میکند
                        \item  سامانه درخواست کاربر را دریافت کرده و
                        و کاربر را از سامانه خارج میکند و اطلاعات لازم را (مثل ساعت خروج و …) ثبت میکند                        
                    \end{enumerate}
                }{کاربر وارد سیستم شده باشد}{کاربر از سامانه خارج شود}{انصراف}
            
        
        \pagebreak
        \zirtitr{نمودارهای موارد کاربرد به تفکیک زیرسیستم}
            \begin{figure}[H]
                \centering
                    \includegraphics[width = 0.30\textwidth]{files/use-case/profile.png}
                \end{figure}
                \begin{figure}[H]
                    \centering
                    \includegraphics[width = 0.20\textwidth]{files/use-case/search.png}
                \end{figure}
                \begin{figure}[H]
                    \centering
                    \includegraphics[width = 0.40\textwidth]{files/use-case/channel.png}
                \end{figure}
                \begin{figure}[H]
                    \centering
                    \includegraphics[width = 0.40\textwidth]{files/use-case/supscription.png}
                \end{figure}
                    \begin{figure}[H]
                    \centering
                \includegraphics[width = 0.40\textwidth]{files/use-case/access.png}
                \caption{نمودارهای موارد کاربرد به تفکیک زیرسیستم}
            \end{figure}
 
    \titr{نمودارهای فعالیت}
        \newcounter{activityid}
\setcounter{activityid}{0}
\newcommand{\activitydiagramofusecase}[3]{
    \stepcounter{activityid}
    \begin{figure}[H]
        \centering
        \includegraphics[width = 0.5\textwidth]{files/figures/ActivityDiagrams/#1}
        \caption{نمودار فعّالیت برای مورد کاربرد «#2» به شناسه‌ی \arabic{activityid}}
        \label{fig:#3}
    \end{figure}
}

\activitydiagramofusecase{1_sign_up.png}{عضویت}{sign_up}
\activitydiagramofusecase{2_login.png}{ورود}{login}
\activitydiagramofusecase{3_channel_create.png}{ساختن کانال}{channel_create}
\activitydiagramofusecase{4_post_create.png}{فرستادن پست}{post_create}
\activitydiagramofusecase{5_plan_define.png}{تعریف تعرفه}{plan_define}
\activitydiagramofusecase{6_admin_set.png}{تعیین مدیر}{admin_set}
\activitydiagramofusecase{7_channel_join.png}{عضو شدن در کانال}{channel_join}
\activitydiagramofusecase{8_subscription_buy.png}{خریدن اشتراک}{subscription_buy}
\activitydiagramofusecase{9_checkout.png}{تسویه‌ی مالی}{checkout}
\activitydiagramofusecase{10_share_set.png}{تعیین سهم مدیران}{share_set}
\activitydiagramofusecase{11_post_view.png}{نمایش پست}{post_view}
\activitydiagramofusecase{12_post_purchase.png}{خرید پست}{post_purchase}
\activitydiagramofusecase{13_channels_list_view.png}{نمایش فهرست کانال‌ها}{channels_list_view}
\activitydiagramofusecase{14_search.png}{جستجو}{search}
\activitydiagramofusecase{16_account_exit.png}{خروج از حساب}{account_exit}


    \titr{کارت‌های \lr{crc}}
        \newcommand{\crccard}[3]{
    \begin{table}[H]
        \centering
        \begin{tabular}{||p{104mm}||}
            \hline
            \begin{center}
                \textbf{#1}
            \end{center}
            \\\hline
        \end{tabular}
        \begin{tabular}{||p{5cm}|p{5cm}||}
            
            مسئولیت‌ها:#2 & همکاران: #3 \\\hline
        \end{tabular}
    \end{table}
}

\crccard{کاربر}{
    \begin{itemize}
        \item پرداختن هزینه‌ی محتوای پولی
        \item مشاهده‌ی فهرست کانال‌های عضوشده
        \item عضو شدن در کانال
        \item خریدن اشتراک
        \item مشاهده‌ی پست‌ها (با توجّه به سطح دسترسی)
    \end{itemize}
}{
    \begin{itemize}
        \item کانال
        \item اشتراک کاربر
        \item تراکنش
        \item پست
    \end{itemize}
}

\crccard{مالک کانال (زیرکلاسی از «مدیر کانال»)}{
    \begin{itemize}
        \item گزیدن مدیر
        \item تعیین سهم مدیران
        \item ایجاد کانال
        \item تعیین تعرفه‌های اشتراک‌ها
    \end{itemize}
}{
    \begin{itemize}
        \item کانال
        \item مدیر کانال
    \end{itemize}
}

\crccard{مدیر کانال}{
    \begin{itemize}
        \item ایجاد محتوا (پست) برای کانال
    \end{itemize}
}{
    \begin{itemize}
        \item کانال
        \item پست
    \end{itemize}
}

\crccard{کانال}{
    \begin{itemize}
        \item نمایش پست‌ها (با توجه به سطح دسترسی)
        \item نمایش مشخصات و جزئیات کانال
        \item دربرگرفتن فهرست اعضا و مدیران
    \end{itemize}
}{
    \begin{itemize}
        \item کاربر
        \item عضو عادی
        \item عضو ویژه
        \item مدیر
        \item پست
        \item جستجو
    \end{itemize}
}

\crccard{جستجو}{
    \begin{itemize}
        \item انجام جستجو
        \item نمایش نتایج جستجو
    \end{itemize}
}{
    \begin{itemize}
        \item کاربر
        \item کانال
    \end{itemize}
}

\crccard{عضو ویژه}{
    \begin{itemize}
        \item مشاهده‌ی پست
        \item نوع اشتراک
    \end{itemize}
}{
    \begin{itemize}
        \item اشتراک
        \item کانال
        \item پست 
        \item کاربر
    \end{itemize}
}

\crccard{عضو عادی}{
    \begin{itemize}
        \item مشاهده‌ی پست
        \item پرداختن موردی
        \item خریدن اشتراک
    \end{itemize}
}{
    \begin{itemize}
        \item کانال
        \item پست
        \item کاربر
    \end{itemize}
}

\crccard{اشتراک کاربر}{
    \begin{itemize}
        \item تاریخ آغاز
        \item مدت زمان (نوع اشتراک)
        \item بررسی منقضی‌شده بودن
    \end{itemize}
}{
    \begin{itemize}
        \item کاربر
        \item کانال
    \end{itemize}
}

\crccard{تعرفه‌ی کانال}{
    \begin{itemize}
        \item ایجاد شدن
        \item فهرست تعرفه‌های کانال
        \item تغییر کردن
    \end{itemize}
}{
    \begin{itemize}
        \item کاربر
        \item مالک کانال
        \item عضو عادی
        \item تراکنش
    \end{itemize}
}

\crccard{پست}{
    \begin{itemize}
        \item ایجاد شدن
        \item رایگان یا پولی بودن
        \item نمایش با توجه به سطح دسترسی
    \end{itemize}
}{
    \begin{itemize}
        \item مدیر کانال
        \item کانال
    \end{itemize}
}

\crccard{پست پولی (زیرکلاسی از «پست»)}{
    \begin{itemize}
        \item قیمت
        \item عنوان پست
        \item خلاصه‌ی پست پولی
        \item نمایش با توجّه به سطح دسترسی
    \end{itemize}
}{
    \begin{itemize}
        \item عضو عادی
        \item کانال
        \item مدیر کانال
        \item عضو ویژه
        \item تراکنش
    \end{itemize}
}

\crccard{تراکنش}{
    \begin{itemize}
        \item ایجاد شدن
        \item محاسبه‌ی درآمد (ماهانه‌ی) کانال
        \item ذخیره‌ی تراکنش
        \item سوابق تراکنش‌ها
        \item تسویه‌ی مالی
    \end{itemize}
}{
    \begin{itemize}
        \item عضو عادی
        \item کانال
        \item پست
        \item مدیر کانال
    \end{itemize}
}

\crccard{عکس}{
    \begin{itemize}
        \item ایجاد شدن
        \item حجم
        \item نمایش
    \end{itemize}
}{
    \begin{itemize}
        \item کانال
        \item عضو عادی
        \item عضو ویژه
        \item مدیر کانال
    \end{itemize}
}

\crccard{فیلم (ویدیو)}{
    \begin{itemize}
        \item ایجاد شدن
        \item نمایش
        \item طول فیلم
        \item حجم فیلم
    \end{itemize}
}{
    \begin{itemize}
        \item کانال
        \item عضو عادی
        \item عضو ویژه
        \item مدیر کانال
    \end{itemize}
}

\crccard{فایل صوتی}{
    \begin{itemize}
        \item ایجاد شدن
        \item پخش شدن
        \item طول صوت
        \item حجم فایل
    \end{itemize}
}{
    \begin{itemize}
        \item کانال
        \item عضو عادی
        \item عضو ویژه
        \item مدیر کانال
    \end{itemize}
}

% \begin{figure}[H]

%     \centering
%         \includegraphics[width = 0.8\textwidth]{files/crc/1.png}
% \end{figure}        
% \begin{figure}[H]
%     \centering
%         \includegraphics[width = 0.8\textwidth]{files/crc/2.png}
% \end{figure}       
% \begin{figure}[H]
%     \centering
%         \includegraphics[width = 0.8\textwidth]{files/crc/3.png}
% \end{figure}       
% \begin{figure}[H]
%     \centering
%         \includegraphics[width = 0.8\textwidth]{files/crc/4.png}
% \end{figure}       
% \begin{figure}[H]
%     \centering
%         \includegraphics[width = 0.8\textwidth]{files/crc/5.png}
% \end{figure}       
% \begin{figure}[H]
%     \centering
%         \includegraphics[width = 0.8\textwidth]{files/crc/6.png}
% \end{figure}       
% \begin{figure}[H]
%     \centering
%         \includegraphics[width = 0.8\textwidth]{files/crc/7.png}
% \end{figure}       
% \begin{figure}[H]
%     \centering
%         \includegraphics[width = 0.8\textwidth]{files/crc/8.png}
% \end{figure}       
% \begin{figure}[H]
%     \centering
%         \includegraphics[width = 0.8\textwidth]{files/crc/9.png}
% \end{figure}       
% \begin{figure}[H]
%     \centering
%         \includegraphics[width = 0.8\textwidth]{files/crc/10.png}
% \end{figure}       
% \begin{figure}[H]
%     \centering
%         \includegraphics[width = 0.8\textwidth]{files/crc/11.png}
% \end{figure}       
% \begin{figure}[H]
%     \centering
%         \includegraphics[width = 0.8\textwidth]{files/crc/12.png}
% \end{figure}       
% \begin{figure}[H]
%     \centering
%         \includegraphics[width = 0.8\textwidth]{files/crc/13.png}
% \end{figure}       
% \begin{figure}[H]
%     \centering
%         \includegraphics[width = 0.8\textwidth]{files/crc/14.png}
% \end{figure}            
% \begin{figure}[H]
%     \centering
%         \includegraphics[width = 0.8\textwidth]{files/crc/16.png}
% \end{figure}       
% \begin{figure}[H]
%     \centering
%         \includegraphics[width = 0.8\textwidth]{files/crc/17.png}
% \end{figure}       







    \titr{معماری سیستم}
        \zirtitr{معماری کلی}
از معماری Service-Oriented Architecture استفاده می‌کنیم این معماری یک رویکرد برای ساخت سیستم‌های نرم‌افزاری است که بر اساس سرویس سازماندهی شده‌اند. در این معماری، هر سرویس به عنوان یک واحد مستقل و قابل استفاده در سیستم تعریف می‌شود که قابل استفاده بودن آن با سرویس‌های دیگر را فراهم می‌کند.

\zirtitr{معماری سرور }
معماری Model-View-Template (MVT) یک الگوی طراحی برای توسعه وب است که در فریم‌ورک Django استفاده می‌شود. این معماری شامل سه بخش اصلی است:

\begin{enumerate}
\item مدل (Model): این بخش شامل داده‌های برنامه و روابط آن‌ها با یکدیگر است. مدل ها به عنوان نقطه شروع برای تعریف داده ها و روابط آن ها در پایگاه داده استفاده می شود.

\item نمایش (View): این بخش شامل کدهای لازم برای پردازش درخواست کاربر و نمایش صفحات وب است. در این قسمت، کدهای لازم برای پاسخ به درخواست های HTTP، پردازش داده های فرستاده شده توسط کاربر، و نمایش صفحات HTML به کاربران تولید می شود.

\item قالب (Template): این بخش شامل قالب های HTML است که به عنوان نقطه خروجی برنامه استفاده می شود. قالب ها حاوی کدهای HTML، CSS و JavaScript هستند که برای نمایش داده ها به کاربران استفاده می شوند.
\end{enumerate}

    \titr{نمودارهای فعالیت با خطوط شنا}
        \newcounter{activityswimlaneid}
\setcounter{activityswimlaneid}{0}
\newcommand{\activitydiagramswimlane}[3]{
    \stepcounter{activityswimlaneid}
    \begin{figure}[H]
        \centering
        \includegraphics[width = 0.5\textwidth]{files/figures/ActivityDiagramsWithSwimlane/#1}
        \caption{نمودار فعّالیت برای مورد کاربرد «#2» به شناسه‌ی \arabic{activityswimlaneid}}
        \label{fig:#3_swimlane}
    \end{figure}
}

\activitydiagramswimlane{1_sign_up.jpg}{عضویت}{sign_up}
\activitydiagramswimlane{2_login.jpg}{ورود}{login}
\activitydiagramswimlane{3_channel_create.jpg}{ساختن کانال}{channel_create}
\activitydiagramswimlane{4_post_create.jpg}{فرستادن پست}{post_create}
\activitydiagramswimlane{5_plan_define.jpg}{تعریف تعرفه}{plan_define}
\activitydiagramswimlane{6_admin_set.jpg}{تعیین مدیر}{admin_set}
\activitydiagramswimlane{7_channel_join.jpg}{عضو شدن در کانال}{channel_join}
\activitydiagramswimlane{8_subscription_buy.jpg}{خریدن اشتراک}{subscription_buy}
\activitydiagramswimlane{9_checkout.jpg}{تسویه‌ی مالی}{checkout}
\activitydiagramswimlane{10_share_set.jpg}{تعیین سهم مدیران}{share_set}
\activitydiagramswimlane{11_post_view.jpg}{نمایش پست}{post_view}
\activitydiagramswimlane{12_post_purchase.jpg}{خرید پست}{post_purchase}
\activitydiagramswimlane{13_channels_list_view.jpg}{نمایش فهرست کانال‌ها}{channels_list_view}
\activitydiagramswimlane{14_search.jpg}{جستجو}{search}
\activitydiagramswimlane{16_account_exit.jpg}{خروج از حساب}{account_exit}

    
    \titr{نمودار کلاس‌های تحلیل}
        \begin{figure}[H]
    \centering
    \includegraphics[width = 0.8\textwidth]{files/figures/ClassDiagram/classDiagram.jpg}
    \caption{نمودار کلاس‌های تحلیل}
    \label{fig:classDiagram}
\end{figure}

\begin{figure}[H]
    \centering
    \includegraphics[width = 0.5\textwidth]{files/figures/ClassDiagram/classCatalogue.jpg}
    \caption[نمودار کلاس‌های کاتالوگ]{نمودار کلاس‌های کاتالوگ - این کلاس‌ها برای مدیریت مجموعه‌ای از اشیاء استفاده می‌شوند که در این نمودار برای سادگی نشان داده نشده‌اند ولی در یادداشت‌هایی بیان شده‌اند.}
    \label{fig:classDiagram}
\end{figure}

    \titr{توالی نمودار}
        % \newcounter{sequenceid}
% \setcounter{sequenceid}{0}
\newcommand{\secuencediagram}[3]{
    % \stepcounter{sequenceid}
    \begin{figure}[H]
        \centering
        \includegraphics[width = 0.5\textwidth]{files/figures/SequenceDiagrams/#1}
        \caption{نمودار توالی برای #2}
        \label{fig:#3_seq}
    \end{figure}
}

\secuencediagram{1_sign_up.jpg}{عضویت}{sign_up}
\secuencediagram{2_login.jpg}{ورود}{login}
\secuencediagram{3_channel_create.jpg}{ساختن کانال}{channel_create}
\secuencediagram{4_post_create.jpg}{فرستادن پست}{post_create}
\secuencediagram{5_plan_define.jpg}{تعریف تعرفه}{plan_define}
\secuencediagram{6_admin_set.jpg}{تعیین مدیر}{admin_set}
\secuencediagram{7_channel_join.jpg}{عضو شدن در کانال}{channel_join}
\secuencediagram{8_subscription_buy.jpg}{خریدن اشتراک}{subscription_buy}
\secuencediagram{9_checkout.jpg}{تسویه‌ی مالی}{checkout}
\secuencediagram{10_share_set.jpg}{تعیین سهم مدیران}{share_set}
\secuencediagram{11_post_view.jpg}{نمایش پست}{post_view}
\secuencediagram{12_post_purchase.jpg}{خرید پست}{post_purchase}
% \secuencediagram{13_channels_list_view.jpg}{نمایش فهرست کانال‌ها}{channels_list_view}
\secuencediagram{14_find_member.jpg}{یافتن عضو}{find_member}


    \titr{نمودار بسته}
        \begin{figure}[H]
    \centering
    \includegraphics[width = 0.8\textwidth]{files/figures/PackageDigarams/packageDiagram.jpg}
    \caption{نمودار بسته}
    \label{fig:packageDiagram}
\end{figure}

    \titr{ نمونه اولیه واسط کاربری }
        \begin{figure}[H]
    \centering
        \includegraphics[width = 0.8\textwidth]{files/ui/signup.png}
        \caption{صفحه‌ی ثبت نام}
\end{figure} 
\begin{figure}[H]
    \centering
        \includegraphics[width = 0.8\textwidth]{files/ui/sameemail.png}
        \caption{صفحه‌ی ثبت نام پس از مواجه با خطای ایمیل تکراری }
\end{figure} 

\begin{figure}[H]
    \centering
        \includegraphics[width = 0.8\textwidth]{files/ui/samephonenumber.png}
        \caption{صفحه‌ی ثبت نام پس از مواجه با خطای شماره تلفن تکراری}
    \centering
        \includegraphics[width = 0.8\textwidth]{files/ui/login.png}
        \caption{صفحه لاگین}
\end{figure} 
% \begin{figure}[H]
%     \centering
%         \includegraphics[width = 0.8\textwidth]{files/ui/nouser.png}
%         \caption{صفحه‌ی لاگین بعد از مواجه با خطای وجود نداشتن نام}
% \end{figure} 
% \begin{figure}[H]
%     \centering
%         \includegraphics[width = 0.8\textwidth]{files/ui/wrongpass.png}
%         \caption{صفحه‌ی لاگین بعد از مواجه با خطای پسورد اشتباه}
% \end{figure} 
\begin{figure}[H]
    \centering
        \includegraphics[width = 0.8\textwidth]{files/ui/channels.png}
        \caption{صفحه‌ی کانال‌ها}
\end{figure} 
  
        
    \titr{واژه‌نامه}
        \begin{table}[H]
    \centering
    \begin{tabular}{|p{4cm}|p{4cm}|p{4cm}|p{4cm}|}
        \hline \textbf{واژه} & \textbf{توضیحات} & \textbf{مترادف} & \textbf{مشابه} \\\hline
 اشتراک & عملیاتی که کاربران برای عضویت در کانال انجام می‌دهند و مبلغی را پرداخت می‌کنند & & \\\hline
 انتساب نقش & تخصیصی نقش به کاربر & & \\\hline
 ایجاد کانال & عملیات ایجاد کانال توسط کاربر & ساخت کانال & \\\hline
 ایمیل & آدرس ایمیل کاربر است & رایانامه & \\\hline
 پست & محتوایی که در کانال به اشتراک گذاشته می‌شود & محتوا & \\\hline
 پست های پولی & پست‌هایی که برای دسترسی به آن‌ها کاربر باید اشتراک تهیه کند یا به صورت موردی بپردازد & محتوای پولی & \\\hline
 تعرفه & مبلغی که برای عضویت در کانال در نظر گرفته می‌شود & & \\\hline
 ثبت نام & فرآیند ثبت نام کاربر در سیستم & & \\\hline
 حساب کاربری & پروفایل شخصی کاربر در سامانه است. & پروفایل کاربری، اکانت & \\\hline
 دسترسی & اجازه دسترسی به محتوا & & \\\hline
 گذرواژه & ترکیب مخفی از کاراکترها که کاربر برای امنیت حساب خود انتخاب می‌کند & رمز (عبور)، کلمه‌ی عبور، پسورد & \\\hline
 زمانبندی & برنامه‌ریزی و زمان‌بندی انجام عملیات‌ها & & \\\hline
 صدا & فایل صوتی & صوت & \\\hline
 عضو ویژه & به هر کاربری که در یک کانال خاص اشتراک دارد، عضو ویژه‌ي آن کانال گفته مي‌شود.& & \\\hline
\end{tabular}
\end{table}
\pagebreak
\begin{table}[H]
    \centering
\begin{tabular}{|p{4cm}|p{4cm}|p{4cm}|p{4cm}|}
 عضويت در كانال& پيوستن به كانال توسط كاربر& & \\\hline
 عنوان& متني كه عنوان محتوا است و موضوع كلي آن محتوا را نشان ميدهد& & \\\hline
 عكس& تصوير و گرافيك& تصوير& \\\hline
 فيلم& ويديو و فيلم& ويديو& \\\hline
 كاربر& به فردی اطلاق می‌شود که با سیستم تعامل دارد& & \\\hline
 كانال& بستري گروهي كه توسط مالك و مدير آن مديريت مي‌شود و ساير كاربران مي‌توانند به آن عضو شده و محتواهاي آن‌ها را مشاهده كنند& & \\\hline
 مالك كانال& كاربري كه نقش مالك را در يك كانال دارد و ايجاد كننده آن كانال است.& & \\\hline
 مبالغ& پول‌ها و مقادیر مالی اطلاق می‌شود که از اشتراک‌ها و محتواهای پرداختی دریافت می‌شود& & \\\hline
 محتوای متنی & متنی که کاربر در کانال به اشتراک می‌گذارد &  &متن \\\hline
 محتوای محدود شده & محتوایی که برای کاربران عادی غیر قابل دسترس است & & \\\hline
 محدودیت دسترسی & محدود بودن دسترسی به محتوا & & \\\hline
 مدیر کانال & کاربری که نقش مدیر را در یک کانال دارد و مسئولیت تولید محتوا در کانال را بر عهده دارد & & \\\hline
 مشاهده پست & مشاهده محتوا های موجود در کانال & دیدن محتوا & \\\hline
 نام کاربری & نامی مستعار برای نمایش نمایه (پروفایل) به دیگران & nickname & نام \\\hline
 ورود & ورود کاربر به حساب کاربری خود & لاگین & \\\hline
 خروج & خروج کاربر از حساب کاربری خود & & \\\hline
 
    \end{tabular}
\end{table}
% \\\hline 
% hline واژه & توضیحات & مترادف\\\hline       
%        \hline \endhead % This is the header for the first page of the table \hline \multicolumn{3}{|r|}{{Continued on next page}}\\\hline       
%        \hline \endfoot % This is the footer for all pages except the last page of the table \hline \endlastfoot % This is the footer for the last page of the table % Now the table content اشتراک & عملیاتی که کاربران برای عضویت در کانال انجام می‌دهند و مبلغی را پرداخت می‌کنند &\\\hline       
%        انتساب نقش & تخصیصی نقش به کاربر &\\\hline       
%        ایجاد کانال & عملیات ایجاد کانال توسط کاربر & ساخت کانال\\\hline       
%        ایمیل & آدرس ایمیل کاربر است & رایانامه\\\hline       
%        پست & محتوایی که در کانال به اشتراک گذاشته می‌شود & محتوا\\\hline 
%        پست های پولی & پست‌هایی که برای دسترسی به آن‌ها کاربر باید اشتراک تهیه کند یا به صورت موردی بپردازد & محتوای پولی\\\hline 
%        تعرفه & مبلغی که برای عضویت در کانال در نظر گرفته می‌شود &\\\hline 
%        ثبت نام & فرآیند ثبت نام کاربر در سیستم &\\\hline 
%        حساب کاربری & پروفایل شخصی کاربر در سامانه است. & پروفایل کاربری، اکانت\\\hline 
%        دسترسی & اجازه دسترسی به محتوا &\\\hline 
%        گذرواژه & ترکیب مخفی از کاراکترها که کاربر برای امنیت حساب خود انتخاب می‌کند & رمز (عبور)، کلمه‌ی عبور، پسورد\\\hline 
%        زمانبندی & برنامه‌ریزی و زمان‌بندی انجام عملیات‌ها &\\\hline 
%        صدا & فایل صوتی & صوت\\\hline 
%        عضو ویژه & به هر کاربری که در یک کانال خاص اشتراک دارد، عضو ویژه‌ي آن کانال گفته مي‌شود &\\\hline 
%        عضويت در كانال & پيوستن به كانال توسط كاربر &\\\hline 
%        عنوان & متني كه عنوان محتوا است و موضوع كلي آن محتوا را نشان ميدهد &\\\hline 
%        عكس & تصوير و گرافيك & تصوير\\\hline 
%        فيلم & ويديو و فيلم & ويديو\\\hline 
%        كاربر & به فردي اطلاق مي‌شود كه با سيستم تعامل دارد &\\\hline 
%        كانال & بستري گروهي كه توسط مالك و مدير آن مديريت مي‌شود و ساير كاربران مي‌توانند به آن عضو شده و محتواهاي آن‌ها را مشاهده كنند &\\\hline 
%        مالك كانال & كاربري كه نقش مالك را در يك كانال دارد و ايجاد كننده آن كانال است. &\\\hline 
%        مبالغ & پول‌ها و مقادير مالي اطلاق مي‌شود كه از اشتراك‌ها و محتواهاي پرداختي دريافت مي‌شود &\\\hline 
%        محتوای متنی & متنی که کاربر در کانال به اشتراک می‌گذارد & متن\\\hline 
%        محتوای محدود شده & محتوایی که برای کاربران عادی غیر قابل دسترس است &\\\hline 
%        محدودیت دسترسی & محدود بودن دسترسی به محتوا &\\\hline 
%        مدیر کانال & کاربری که نقش مدیر را در یک کانال دارد و مسئولیت تولید محتوا در کانال را بر عهده دارد &\\\hline 
%        مشاهده پست & مشاهده محتوا های موجود در کانال & دیدن محتوا\\\hline 
%        نام کاربری & نام منحصر به فردی که کاربر برای شناسایی خود در سامانه انتخاب می‌کند &\\\hline           
%        ورود & فرآیند ورود کاربر به سیستم &\\\hline
%        خروج & فرآیند خروج کاربر به سیستم & \\\hline
%        \end{tabular} 
    
    \titr{برنامه‌ی زمان‌بندی فاز ۳}
        \begin{figure}[H]
            \centering
            \includegraphics*[width = 0.7\textwidth]{files/figures/phase3timing/page1.png}
            \includegraphics*[width = 0.7\textwidth]{files/figures/phase3timing/page2.png}
            \caption{برنامه‌ی زمانبندی‌شده‌ی فاز ۳}
        \end{figure}
    \titr{چک‌لیست خارج‌شده از اسلایدها}
        \zirtitr{چک‌لیست مربوط به کارت‌های CRC}
\begin{figure}[H]
    \centering
    \includegraphics[width = 0.8\textwidth]{files/figures/checklists/crc.png}
    \caption{چک‌لیست مربوط به کارت‌های CRC}
    \label{fig:crccl}
\end{figure}

\zirtitr{چک‌لیست مربوط به مديريت ريسكها}
\begin{figure}[H]
    \centering
    \includegraphics[width = 0.8\textwidth]{files/figures/checklists/riskManagement.png}
    \caption{چک‌لیست مربوط به مديريت ريسكها}
    \label{fig:rmcl}
\end{figure}

\zirtitr{چک‌لیست مربوط به نمودارهای فعالیت}
\begin{figure}[H]
    \centering
    \includegraphics[width = 0.8\textwidth]{files/figures/checklists/activityDiagrams.png}
    \caption{چک‌لیست مربوط به نمودارهای فعالیت}
    \label{fig:adcl}
\end{figure}

\zirtitr{چک‌لیست مربوط به \lr{executable architectural baseline}}
\begin{figure}[H]
    \centering
    \includegraphics[width = 0.8\textwidth]{files/figures/checklists/activityDiagrams.png}
    \caption{چک‌لیست مربوط به \lr{executable architectural baseline}}
    \label{fig:eabcl}
\end{figure}
            % \begin{figure}[H]
    %     \centering
    %     \includegraphics[width = 0.5\textwidth]{files/9.png}
    %     \caption{
    %         تعیین ورودی‌های eprom در پروتئوس
    %     }
    %     % \label{fig:transition_diagram_q6_p2}
    % \end{figure}

    % \begin{figure}[H]
    %     \centering
    %     \includegraphics[width = 0.4\textwidth]{files/1.png}\hfill
    %     \includegraphics[width = 0.4\textwidth]{files/2.png}\\
    %     \includegraphics[width = 0.4\textwidth]{files/3.png}\hfill
    %     \includegraphics[width = 0.4\textwidth]{files/4.png}\\
    %     \includegraphics[width = 0.4\textwidth]{files/5.png}\hfill
    %     \includegraphics[width = 0.4\textwidth]{files/6.png}\\
    %     \includegraphics[width = 0.4\textwidth]{files/7.png}\hfill
    %     \includegraphics[width = 0.4\textwidth]{files/8.png}\\
    %     \caption{
    %         نتایج تولید اعداد فیبوناچی
    %     }
    %     % \label{fig:transition_diagram_q6_p2}
    % \end{figure}
\end{document}